\documentclass[11pt]{article}

\usepackage[dvips]{graphicx}
\usepackage{rotating}

\usepackage{freerbmt09}
\usepackage[utf8x]{inputenc}
\usepackage{times}
\usepackage{natbib}
\usepackage{url}
\usepackage{latexsym}

\title{Implementing an efficient and scalable Machine Translation service}

\author{Pasquale Minervini\\
  Department of Computer Science \\
  Nonesuch State University \\
  Utopia, NS 12345 \\
  {\tt jane.doe@cs.nsu.edu} \And
  John Smith \\
  Department of Linguistics \\
  Another State University \\
  Collegetown, AS 98765 \\  
  {\tt jsmith@ling.asu.edu}}

\date{}

\begin{document}

\maketitle

\begin{abstract}
Service Oriented Architecture (SOA) is a paradigm for organizing and utilizing distributed services that may be under the 
control of different ownership domains and implemented using various technology stacks. In some contexts, an organization
using an IT infrastructure implementing the SOA paradigm can take a great benefit from an efficient Rule-Based Machine 
Translation (RBMT) service. This paper describes the architecture used to develop an efficient, scalable and easy to integrate
in new and existing business processes RBMT service; the service is based on Apertium, a free/open-source machine translation platform.
\end{abstract}

\section{Introduction}

Service Oriented Architecture is an architectural paradigm providing  a set of principles of governing concepts used during phases 
of systems development and integration. In such an architecture functionalities are packaged as interoperable, loosely coupled
services that may be used to build infrastructures enabling those with needs (consumers) and those with capabilities (providers) 
to interact across different domains of technology and ownership.

Several new trends in the computer industry rely upon SOA as the enabling foundation, including the automation of Business Process 
Management (BPM) and the multitude of new architecture and design patterns generally referred to as Web 2.0~\cite{web20}.\\

In some contexts, an organization could take a great benefit by integrating a MT service in its IT infrastructure to overcome 
language barriers; for example, a common problem arising when building a Knowledge Base starting from a corpus of free text for
a certain domain (like biomedical, financial etc.) is that the text isn't always in a language that can be comprehended by the
domain experts and/or the knowledge extraction tools beign used.

A possible solution to this problem consists in using a Machine
Translation service (eventually using domain-specific dictionaries, rules etc.) to translate the free text from a language to
another with an high accuracy, so that it's then possible to start extracting knowledge from it.\\

{\bf Example:} MetaMap~\cite{metamap} is an application that allows mapping text to UMLS\textregistered 
Metathesaurus\textregistered\footnote{The UMLS\textregistered Metathesaurus\textregistered, the largest thesaurus in the biomedical domain, 
provides a representation of biomedical knowledge consisting of concepts classified by semantic type and both hierarchical and non-hierarchical 
relationships among the concepts. ~\cite{umls}} concepts, which have proved to be useful for many applications, including decision support 
systems, management of patient records, information retrieval and data mining within the biomedical domain.

\begin{figure}[!ht]
\begin{center}
\includegraphics[width=7.5cm]{mtsoa}
\end{center}
\caption{Business Process Model}
\label{fig:mtsoa}
\end{figure}

Currently MetaMap is only available for English free text, which makes difficult the use of UMLS\textregistered 
Metathesaurus\textregistered to represent concepts from biomedical texts written in languages different than English.
A possible way to overcome this limitation consists in using RBMT techniques (possibly with dictionaries, translation rules, 
lexical selection techniques etc. specific for the biomedical domain) to translate the free text from its original language 
to English, and then process it as in Figure \ref{fig:mtsoa}.\\

We realized a prototype service relying on Apertium~\cite{armentano05p}, a free/open-source machine translation platform being 
developed with funding from the Spanish government and the government of Catalonia at the Universitat d'Alacant (University of Alicante), 
for its translation capabilities, and on N-Gram Based Text Categorization~\cite{textcat} for language detection.

\section{Exposing the Service}

Service's interface should allow to access to the following capabilities:

\begin{description}
  \item[translate], for the automatic translation of free text from a source language to a destination language;
  \item[detect], for automatic language guessing;
\end{description}

In SOA, interoperability between services is achieved by using standard languages for the description of service interfaces and the communications
among services. A widely accepted technology for implementing SOA consists in using a technology called Web Services~\cite{soa}, defined by the W3C
as defined by the W3C as ``a software system designed to support interoperable machine-to-machine interaction over a network. It has an interface 
described in a machine-processable format (specifically WSDL). Other systems interact with the Web service in a manner prescribed by its description 
using SOAP-messages, typically conveyed using HTTP with an XML serialization in conjunction with other Web-related standards.''~\cite{wsgloss}. An
alternative to SOAP is XML-RPC~\cite{xmlrpcspec}, a remote procedure call protocol which uses XML to encode its calls and HTTP as a transport mechanism.\\

\section{Service's Internals}



\begin{figure}[!ht]
\begin{center}
\includegraphics[width=7.5cm]{resource_pool}
\end{center}
\caption{Object Pool}
\label{fig:rp}
\end{figure}

\section*{Acknowledgements}

Development for this project was funded as part of the Google Summer of Code\footnote{\url{http://code.google.com/soc/}} programme. In addition, I'm
grateful to Jimmy O'Regan, Francis Tyers and others in The Apertium Project for constant help.


\bibliographystyle{apalike}
\bibliography{freerbmt09}

\end{document}
