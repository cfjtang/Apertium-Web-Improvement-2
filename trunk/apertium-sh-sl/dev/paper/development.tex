\section{Development}

\subsection{Resources}
This language pair was developed with the
aid of on-line resources containing word definitions and flective
paradigms, such as \emph{Hrvatski jezični
  portal}\footnote{\url{http://hjp.srce.hr}} for the BCMS side. For
the Slovene side we used a similar online resource \emph{Slovar
  slovenskega knjižnega
  jezika},\footnote{\url{http://bos.zrc-sazu.si/sskj.html}}, and the
\emph{Amebis Besana} flective
lexicon.\footnote{\url{http://besana.amebis.si/pregibanje/}}

The bilingual dictionary for the language pair was developed from scratch,
using the \emph{EUDict}\footnote{\url{http://eudict.com/}} online
dictionary and \emph{Google
  Translate}\footnote{\url{http://translate.google.com/}}.

\subsection{Morphological analysis and generation}
The basis for this language pair are the morphological
lexicons for BCMS (from
the language pair BCMS-Macedonian, {\small{\tt apertium-bhs-mk}}) and Slovene (from the
language pair Slovene-Spanish, {\small{\tt apertium-sl-es}}). Both
lexicons are written in the XML formalism of
\emph{lttoolbox}\footnote{\url{http://wiki.apertium.org/wiki/Lttoolbox}}
(\citealp{rojas2005construccion}), and were developed as parts of
their respective language pairs, during the Google Summer of Code
2011.\footnote{\url{http://code.google.com/soc/}}. Since the lexicons
had been developed using different frequency lists, and slightly
different tagsets, they have been further trimmed and updated to
synchronise their coverage.

%wiktionaries and Wikipedia, as well as an SETimes corpus\footnote{\url{http://opus.lingfil.uu.se/SETIMES.php}} (\citealp{tyers2010south}) and a%corpus composed from the Serbian, Bosnian, Croatian and Serbo-Croatian Wikipedias.

%% Other resources for morphological analysis of Serbian and Croatian exist
%% (\citealp{vitas2004intex}, \citealp{vitas2003processing}, \citealp{agic2008improving}, \citealp{snajder08automatic}), 
%% to our knowledge there are none freely available for either Serbian, Bosnian or
%% Croatian. 

\subsection{Disambiguation}
Though for both languages there exists a number of tools for
morphological taggin and disambiguation (\todo{reference some}), there
are none freely available. Likewise, the adequatly tagged corpora are
mostly non-free (\todo{reference some non-free corpora, like
  1984.,...}). \todo{See on the taggers/corpora for Slovene}\todo{See
  on the taggers/corpora for BCMS}.
Since both BCMS Slovene are highly flective languages, the
automatically trained tagger would be inefficient blabla, we chose to omit the
statistical tagger canonically used in Apertium language pairs, and
use solely Constraint Gramar (CG) for disambiguation. The CG module
does not provide complete disambiguation, so in the case of any
remaining ambiguity the system picks the first output analysis.

\todo{Disambiguation examples}

\subsection{Lexical transfer}
The lexical transfer was done with an \emph{lttoolbox} letter
transducer composed of bilingual dictionary entries. Additional
paradigms were added to the transducer to compensate for the tagset
notational differences.

\subsection{Lexical selection}
\todo{Lexical selection examples}

\todo{use some sh\_HR and sh\_SR examples perhaps 'univerza' $\rightarrow$ \{univerzitet,sveučilište\}}

\subsection{Transfer}
\todo{describe apertium transfer}
The BCMS and Slovene languages are very closely related, and their
morphologies are extremely similar. Most of non-technical transfer
rules are thus written only for rare syntactic differences. These are
mostly about clitic ordering, and different noun case usage.

\todo{Transfer examples, esp. the differences}
