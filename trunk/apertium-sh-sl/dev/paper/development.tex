\section{Development}

\subsection{Resources}
This language pair was developed with the
aid of on-line resources containing word definitions and flective
paradigms, such as \emph{Hrvatski jezični
  portal}\footnote{\url{http://hjp.srce.hr}} for the BCMS side. For
the Slovene side we used a similar online resource \emph{Slovar
  slovenskega knjižnega
  jezika},\footnote{\url{http://bos.zrc-sazu.si/sskj.html}}, and the
\emph{Amebis Besana} flective
lexicon.\footnote{\url{http://besana.amebis.si/pregibanje/}}

The bilingual dictionary for the language pair was developed from scratch,
using the \emph{EUDict}\footnote{\url{http://eudict.com/}} online
dictionary and \emph{Google
  Translate}\footnote{\url{http://translate.google.com/}}.

\subsection{Morphological analysis and generation}
The basis for this language pair are the morphological
lexicons for BCMS (from
the language pair BCMS-Macedonian, {\small{\tt apertium-bhs-mk}}) and Slovene (from the
language pair Slovene-Spanish, {\small{\tt apertium-sl-es}}). Both
lexicons are written in the XML formalism of
\emph{lttoolbox}\footnote{\url{http://wiki.apertium.org/wiki/Lttoolbox}}
(\citealp{rojas2005construccion}), and were developed as parts of
their respective language pairs, during the Google Summer of Code
2011.\footnote{\url{http://code.google.com/soc/}}. Since the lexicons
had been developed using different frequency lists, and slightly
different tagsets, they have been further trimmed and updated to
synchronise their coverage.

%wiktionaries and Wikipedia, as well as an SETimes corpus\footnote{\url{http://opus.lingfil.uu.se/SETIMES.php}} (\citealp{tyers2010south}) and a%corpus composed from the Serbian, Bosnian, Croatian and Serbo-Croatian Wikipedias.

%% Other resources for morphological analysis of Serbian and Croatian exist
%% (\citealp{vitas2004intex}, \citealp{vitas2003processing}, \citealp{agic2008improving}, \citealp{snajder08automatic}), 
%% to our knowledge there are none freely available for either Serbian, Bosnian or
%% Croatian. 

\subsection{Disambiguation}
Though for both languages there exists a number of tools for
morphological taggin and disambiguation (\todo{reference some}), there
are none freely available. Likewise, the adequatly tagged corpora are
mostly non-free (\todo{reference some non-free corpora, like
  1984.,...}). \todo{See on the taggers/corpora for Slovene}\todo{See
  on the taggers/corpora for BCMS}.  Since both BCMS Slovene are
highly inflected languages, the automatically trained statistical
tagger canonically used in Apertium language pairs would not give
satisfactory results. For this reason we chose to use solely
Constraint Gramar (CG) for disambiguation. The CG module does not
provide complete disambiguation, so in the case of any remaining
ambiguity the system picks the first output analysis.

Due to the similarities between the languages, we were able to
reuse much of the rules developed earlier for BCMS. Following are
some examples of disambiguation rules:

\todo{napomenut u zaključku da su pravila uglavnom rudimentarna}

\begin{itemize}

\item Simple adverb vs. adjective rule:

\sentenceexample{
Ja često jedem ribu. $\leftrightarrow$ Jaz pogosto jedem ribo.

[I{\sc.prn.nom}] [often{\sc.adv}] [eat{\sc.verb}] [fish{\sc.n.acc}]

(I eat fish often.)
}

For this phrase the morphological analyser gives:

{\small
\begin{Verbatim}
"<Ja>"
    "free" prn pers p1 mfn sg nom 
"<često>"
    "često" adv
;   "čest" adj pst nt sg nom ind
;   "čest" adj pst nt sg nom def
;   "čest" adj pst nt sg voc ind
;   "čest" adj pst nt sg voc def
;   "čest" adj pst nt sg acc ind
;   "čest" adj pst nt sg acc def
"<jedem>"
    "jesti" vb imperf tv pres p1 sg
;   "jesti" vb imperf ref pres p1 sg
;   "jesti" vb imperf iv pres p1 sg
"<ribu>"
     "riba" n f sg acc 
\end{Verbatim}
}

The rule used to disambiguate the adverb \emph{često} in this phrase

{\small\begin{verbatim}
    SELECT Adv IF 
        (0 Adv OR A) 
        (1C V)
\end{verbatim}
}

selects the adverb reading in an adverb/adjective ambiguity if the
word following is unambiguously a verb.

\item Preposition based case disambiguation:

\sentenceexample{
Za našo ljubo staro mater. $\leftrightarrow$ Za našu dragu staru majku.

[For{\sc.pr.acc}] [our{\sc.prn.acc}] [dear{\sc.prn.acc}] [old{\sc.prn.acc}] [mother{\sc.n.acc}]

(For our dear old mother.)
}

Noun phrases in both languages typically generate a great number of
ambiguities.

{\small
\begin{Verbatim}
    "<Za>"
         "za" pr acc 
    ;    "za" pr gen
    ;    "za" pr ins
    "<našo>"
         "naš" prn pos p1 f sg acc 
    ;    "naš" prn pos p1 f sg ins
    "<ljubo>"
         "ljub" adj f sg acc ind 
    ;    "ljubo" adv sint
    ;    "ljub" adj nt sg nom ind
    ;    "ljub" adj nt sg acc ind
    ;    "ljub" adj f sg ins ind
    "<staro>"
         "star" adj f sg acc ind 
    ;    "staro" adv sint
    ;    "star" adj f sg ins ind
    ;    "star" adj nt sg nom ind
    ;    "star" adj nt sg acc ind
    "<mater>"
         "mati" n f sg acc 
    ;    "mati" n f du gen
    ;    "mati" n f pl gen
\end{Verbatim} 
}

First the rule

{\small
\begin{Verbatim}
    REMOVE Pr + $$Case IF 
        (1 Nominal - $$Case)
\end{Verbatim}
}

removes a case reading from a preposition if it is not followed by an
adjective, noun or a pronoun in the same case, and then the rule

{\small
\begin{Verbatim}
    REMOVE Nominal + $$Case IF
        (NOT -1 Prep + $$Case) 
        (NOT -1 Modifier + $$Case)
\end{Verbatim}
}

\item Noun heuristics:

This simple heuristic selects a genitive reading of a noun after a noun.

{\small
\begin{Verbatim}
    SELECT: N + Gen IF (-1 N)
\end{Verbatim}
}
\end{itemize}



\subsection{Lexical transfer}
The lexical transfer was done with an \emph{lttoolbox} letter
transducer composed of bilingual dictionary entries. Additional
paradigms were added to the transducer to compensate for the tagset
notational differences.


\todo{use some sh\_HR and sh\_SR examples perhaps 'univerza' $\rightarrow$ \{univerzitet,sveučilište\}}

\subsection{Transfer}

The BCMS and Slovene languages are very closely related, and their
morphologies are extremely similar. Most of non-technical transfer
rules are thus written only for rare syntactic differences. These are
mostly about clitic ordering, and different noun case usage.

Following are examples of transfer rules, which also illustrate some
contrastive characteristics of the languages:

\note{Pravila funkcioniraju u oba smjera pa objasniti oba smjera}

\begin{itemize}
\item The future tense:
\enumsentence{
Ja ću gledati\footnote{The encliticised future tense forms (gledat ću / gledaću) are handled equally.} $\rightarrow$ Jaz bom gledal

[I] [will{\sc.clt.p1.sg}] [watch{\sc.inf}] $\rightarrow$ [I] [will{\sc.clt.p1.sg}] [watch{\sc.pres.lp.m.sg}]

(I will watch.)
}
\end{itemize}

\todo{je treba $\rightarrow$ treba}

\todo{ni bilo treba $\rightarrow$ nije trebalo}

\todo{još jedan s treba}

\todo{clitic ordering}

\todo{lahko $\rightarrow$ može primjer 1}

\todo{lahko $\rightarrow$ može primjer 2}

