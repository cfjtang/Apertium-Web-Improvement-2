\subsection{Lexical transfer}
The lexical transfer was done with an \emph{lttoolbox} letter
transducer composed of bilingual dictionary entries. Additional
paradigms were added to the transducer to compensate for the tagset
notational differences.

Since there was no adequate and free Slovene -- Serbo-Croatian parallel corpus, 
we chose to do the lexical selection relying only on Apertium's lexical selection module.
For cases not covered by our hand-written rules, the system would choose the lexical 
default from the bilingual dictionary.

The following are examples of lexical selection rules.

Phonetics based lexical selection:
{\small
\begin{Verbatim}
<rule>
    <match lemma="borza" tags="n.*">
        <select lemma="burza" tags="n.*"/>
    </match>
</rule>
\end{Verbatim}
}

for Croatian, and 
 <rule><match lemma="borza" tags="n.*"><select lemma="berza" tags="n.*"/></match></rule>
for Serbian.

Namely, the Serbian word for "stock exchange" is berza, while in the Croatian language the "burza" is used.
Similar rules were written for the forms for porcelain (procelan in Serbian and porculan in Croatian), 
salt (so and sol) etc.

While the Serbian dialect accepts the Ekavian and Ikavian reflexes, 
the Croatian dialect uses only the Ijekavian reflex.
Since the selection for the different reflexes of the yat vowel is done in the generation process,
no rules were needed in the lexical selection module.

Other than the yat reflex, examples of morhology based rules include:
 <rule><match lemma="točno" tags="adv.*"><select lemma="točno" tags="adv.*"/></match></rule>
for Croatian, and
 <rule><match lemma="točno" tags="adv.*"><select lemma="tačno" tags="adv.*"/></match></rule>
for Serbian.


Internationalisms have been introduced to Croatian and Bosnian mainly through the Italian and German language
whereas they have entered Serbian through French and Russian. 
As a result, the three dialects have developed different phonetic patterns for internatonal words.

Examples of rules for covering such varieties include:
 <rule><match lemma="Betlehem" tags="np.*"><select lemma="Betlehem" tags="np.*"/></match></rule>
for Croatian, and
 <rule><match lemma="Betlehem" tags="np.*"><select lemma="Vitlejem" tags="np.*"/></match></rule>
for Serbian.

Finally, the Croatian months used for the Gregorian calendar have Slavic-derived names and differ from the original Latin names.
For example, the Croatian language has the word siječanj for January, and 
the Serbian language has the word Januar.
These differences were also covered by the lexical selection module



