

\achapter{Continuing On}{Christine S. Lipubs}


This will be the second numbered chapter of the book.  The following
will help to fill up some space.



\section{Some Stuff Again}


One can do definitions, theorems, and\index{theorems} so forth, as
follows.

\begin{definition}
$(A,C,\cdot)$ is a {\em $X$-form system\/} if $A$ is
a class, $C:A\to PX$ and for each $a\in A$, if $\sigma:Ca\to X$ then
$\sigma\cdot a \in A$ such that
    \begin{itemize}
    \item $C(\sigma\cdot a) = \{\sigma x\mid x\in Ca\}$,
    \item $\sigma\cdot a  = a $ if $\sigma x = x$ for $x\in Ca$,
    \item $\sigma'\cdot(\sigma\cdot a ) = (\sigma'\circ\sigma)\cdot a$
          if  $\sigma':C(\sigma\cdot a) \to X$
    \end{itemize}
\end{definition}

\begin{definition}
$(A,C,\cdot) $ is a {\em (elementary) universe} if it is an $A$-form system.
\end{definition}

\begin{theorem}\label{universe-existence-theorem}
For every ontology\index{universe!well-founded} $U$ there is a wf
(af) $U$-universe.  Moreover it is unique up to isomorphism.
\end{theorem}

\begin{proof}
Well, the proof is left to the reader.  Just follow the definitions
and hack it out.
\end{proof}

\noindent
Besides definitions and theorems, you can have axioms, lemmas,
propositions, examples, and other types of formal
mathematical statements.



\section{Some Other Stuff Again}


After {\LaTeX}ing these files, the indexing information will be recorded
in a file \texttt{book.idx}.  This index data\index{index!flags} can be
massaged into something\index{index!makeindex} that can later be
included as an index, using the \texttt{makeindex} program (ask your
local \LaTeX\ expert about it) plus some hands-on work.  The \LaTeX\
manual explains how this works.

Chapter bibliographies will be produced as References\index{References}
sections by {\LaTeX}'s BibTeX\index{bibliographies} program in cooperation
with the \texttt{chapterbib.sty} and \texttt{natbib.sty} packages.
Check out the corresponding comments in \texttt{example.tex} and
\texttt{natbib.summary} along with the \LaTeX\ manual.
For example, we now cite \citealt{krip:nami72}, which is completely
irrelevant here but demonstrates the bibliography citation mechanism.
If nothing else were cited then this chapter's References section would
contain just that one entry.  Additional entries are included in this
chapter's References without being explicitly cited in the text, thanks
to the \texttt{\backslash nocite\braced{\,}} command.
The file \texttt{example-ch02/ch02.bib} has many possible
bibliography entries, but only those that are cited (or ``nocited'')
will appear in the References section at the and of this chapter.
 \nocite{quin:vari60}
 \nocite{putn:real83}
 \nocite{kapl:demo89}
 \nocite{freg:thou77}

Also, an example of a figure is included here.  Figures will appear
at the tops of pages, and the caption is placed below the figure.
Tables also appear at the tops of pages, but the caption is placed
above the body of the table.

 \begin{figure}[t]
 \begin{center}
 \begin{picture}(200,50)(0,0)
 \put(25,25){\circle{20}}
 \put(30,25){\circle{20}}
 \put(35,25){\circle{20}}
 \put(45,25){\circle{20}}
 \put(60,25){\circle{20}}
 \put(63,25){\vector(4,3){16}}
 \put(63,25){\vector(3,4){16}}
 \put(63,25){\vector(3,-1){36}}
 \put(63,25){\vector(3,1){16}}
 \put(15,5){\line(1,0){175}}
 \put(15,5){\line(6,1){175}}
 \put(125,30){\circle*{4}}
 \put(150,35){\oval(60,30)}
 \put(115,25){\dashbox(40,10){}}
 \end{picture}
 \end{center}
 \caption[A Bunch of Overlapping Circles]
	 {A Bunch of Overlapping Circles and Some \\
	  Other Stuff: An Example of a Figure with \\
	  A Long Caption}
 \end{figure}



\section{Typewriter characters in \protect\textbt{cslipubs-extra.sty}}


\index{typewriter characters}

The package \texttt{cslipubs-extra.sty} also includes
some characters normally unavailable in the typewriter
fonts of \texttt{\backslash texttt\braced{\,}} or
{\ttfamily \backslash ttfamily}.  Any \texttt{\$}math\texttt{\$} mode
meanings of the following commands are unaffected.
Yes, \texttt{\backslash sim} and \texttt{\backslash tilde}
are redundant on purpose.

\smallskip
\begin{center}
\renewcommand{\arraystretch}{1.2}
\begin{tabular}{lccc}
Command                      &Character          &Example
                                    &(\texttt{\$}math\texttt{\$} mode) \\[.5ex]
\texttt{\backslash backslash}&\texttt{\backslash}&\texttt{\backslash /}
                                                      &( $\backslash /$ ) \\
\texttt{\backslash caret}    &\texttt{\caret}    &\texttt{10\caret 2}
                                                      &\smaller none \\
\texttt{\backslash lbrace}   &\texttt{\lbrace}   &\texttt{\lbrace 100}
                                                      &( $\lbrace 100$ ) \\
\texttt{\backslash rbrace}   &\texttt{\rbrace}   &\texttt{100\rbrace}
                                                      &( $100\rbrace$ ) \\
\texttt{\backslash sim}      &\texttt{\sim}      &\texttt{\sim 1}
                                                      &( $\sim 1$ ) \\
\texttt{\backslash tilde}    &\texttt{\tilde}    &\texttt{\tilde 1}
                                                      &( $\tilde 1$ ) \\
\end{tabular}
\end{center}
\smallskip



% The References section for this chapter is produced in ../example-ch02.tex

