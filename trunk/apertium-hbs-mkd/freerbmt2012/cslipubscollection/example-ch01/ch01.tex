

\achapter{Getting Started}{Gnu Who}


This is the first numbered chapter of the book.  The following
text illustrates what \texttt{cslipubs.sty} produces, along with
examples of some of the macros in \texttt{cslipubs-extra.sty} and
\texttt{lingmacros.sty}.  This sampling is not exhaustive.



\section{Format of Documents}


Please look at \texttt{example.tex}, which produced this
documentation by using the \texttt{cslipubs.sty} package.
Note that the \texttt{cslipubs.sty} package relies on
Melchior Franz's \texttt{crop.sty} package, a copy of
which accompanies this example.



\section {\protect\textbt{D\protect\smaller RAFT} vs.
          \protect\textbt{\protect\smaller FINAL}}


\index{DRAFT option@\texttt{[DRAFT]} option}
\index{FINAL option@\texttt{[FINAL]} option}

In specifying \texttt{\backslash usepackage\braced{cslipubs}},
the options \texttt{\smaller DRAFT} and \texttt{\smaller FINAL}
are available.  The default is \texttt{\smaller DRAFT}, which
is recommended until the book is almost ready for press:

   \smallskip
   \texttt{\backslash usepackage[{\smaller DRAFT}]\braced{cslipubs}}
   \smallskip

\noindent
The \texttt{\smaller DRAFT} option provides the following:

\begin{itemize}
\item  Each page is centered on $8 {1 \over 2}{}'' \times 11''$
       (USA's standard) paper.

\item  \texttt{\backslash NOTE\braced{\,}} may be used for marginal
       draft comments.\index{note@\texttt{\backslash NOTE\braced{\,}}}
       \NOTE{This marginal note demonstrates the
             \texttt{\backslash NOTE\braced{\,}}
             command, which is meant for reminders
             while using the \texttt{[{\smaller DRAFT}]} option.}

\item  The date is printed at the top outer corner of each page.
\end{itemize}

\noindent
When the book is almost ready for press, change the above
\texttt{\backslash usepackage} command to:

   \smallskip
   \texttt{\backslash usepackage[{\smaller FINAL}]\braced{cslipubs}}
   \smallskip

\noindent
The \texttt{\smaller FINAL} option has the following effects:

\begin{itemize}
\item  Each page is pushed to the top left corner on a printout
       (unless a \texttt{\backslash crop[\,]} command is given, which
       centers each page).

\item  \texttt{\backslash NOTE\braced{\,}} is nullified, producing
       no comments.
\end{itemize}



\section{Some Stuff From \protect\textbt{lingmacros.sty}}


For instance, the following manner of displaying
sentences\index{displayed sentences} or other kinds of examples of
things was originally designed for linguistics documents, and we have
found it to be generally useful in other sorts of documents as well.
  \enumsentence{Peas\index{peas} porridge\index{porridge} hot.  Peas
                porridge cold.\label{peas}}
And then you can have enumerated items within such an enumerated
example.

 \eenumsentence{\item \shortex{3}
        {kalk-n & apra & kpa-ra}
        {sago pudding V SG-OBL & plate VII PL & big-VII PL}
        {`big plates of sago pudding'}
\toplabel{sago}
 \item \shortex{1}
        {pia-ka-tim{\'{\i}}}
        {words O-1SG A-say}
        {`I talked.'}
 \item \shortex{1}
        {na-mpu-wap{\'a}t-ncut}
        {3SG O-3PL A-climb-RM PAST}
        {`They climbed it (the tree).'}
 \item \shortex{2}
        {nan-{\'a}wkura-na & amtra}
        {PL IMP-gather-IMP & food V PL}
        {`Collect food!'}}

The enumeration\index{enumeration} is taken care of automatically.
And you can refer to such enumerated examples without having to keep
track of or otherwise know which numbers they are assigned using
the standard \LaTeX\ \texttt{\backslash label\braced{\,}} command or
the non-standard \texttt{\backslash toplabel\braced{\,}} command.
Look into the
on-line file for this chapter to see how you are able to refer
blindly to examples (\ref{peas}), (\ref{sago}).  A little bit
extra to make sure of enough lines.



\section{Some More Stuff}


You can also produce attribute-value\index{AVMs@{\protect\smaller AVM}s}
matrices, like this:
 \begin{center}
   \avm{alpha & beta \\
        gamma & delta \\
        epsilon & zeta}
 \end{center}
And various kinds of proof-trees are\index{proof trees}
possible,\footnote{Here is a footnote.  Getting these proof trees to
look good takes a lot of careful attention.} though\index{footnotes}
there is a limit to how deeply these things can be embedded:

 \[
 \ddctn{A\quad \ddctn{B \land C}{D}}{E}
 \]

However, you might also want to look at the separate \texttt{avm.sty} style
file by Christopher Manning.

If you want\index{paragraph indentation} the first line of a block of
text following such a display not to be indented, such as when the
display occurs in the middle of a paragraph, then write
\texttt{\backslash noindent} first thing at the start of the block of text.

A table is included here as well.  Figures will appear at the tops of
pages, and the caption is placed below the figure.  Tables also appear
at the tops of pages, but the caption is placed above the body of the
table.

 \begin{table}[t]
 \caption[Means, Medians, and Ranges (Experiment 2)]{ \\
 Means, Medians, and Ranges of Twenty-four Correlations \\
 Computed on Mean Ranks of Students' Choices Between \\
 the Reference, Schelling, and Salience Tasks \\
 \scriptsize (Experiment 2)}\label{mmr}
 \begin{center}
 {\footnotesize\advance\tabcolsep by.33em
 \begin{tabular}{lccc}\hline\\[-2ex]
 \sc Pairs of tasks& \sc Mean& \sc Median& \sc Range\\[.5ex]
 \hline \\[-2ex]
 Reference and Schelling tasks& .80& .89& 0.33--1.00\\
 Reference and salience tasks& .80& .83& 0.27--1.00\\
 Schelling and salience tasks& .84& .89& 0.21--1.00\\[.5ex]
 \hline
 \end{tabular}}
 \end{center}
 \end{table}


And we really need to have enough stuff to take us onto another page.

In the late autumn\index{autumn} of 1903, Professor
R. Blondlot\index{Blondlot}, head of the
Department of Physics at the University of Nancy, member of the French
Academy, and widely known as an investigator, announced the discovery
of a new ray, which he called N ray, with properties far transcending
those of the x-rays.  Reading of his remarkable experiments, I
attempted to repeat his observations, but failed to confirm them after
wasting a whole morning. \dots

Fuel was added by a score of other investigators. Twelve papers had
appeared in the ``Comptes rendus'' before the year was out. \dots

By early summer Blondlot had published twenty papers,
Charpentier\index{Charpentier} twenty, and J.
Becquerel\index{Becquerel} ten, all describing new properties and
sources of the rays.

Scientists in all other countries were frankly skeptical, but the
French Academy stamped Blondlot's\index{Blondlot} work with its
approval by awarding him the Lalande prize of 20,000 francs and its
gold medal `for the discovery of the N rays.'

[Excerpts from ``N rays'' by R. W. Wood,\index{Wood} in R. L.
Weber,\index{Weber} \textit{A Random Walk in Science} (The Institute of
Physics, London) 1973.  The article was, itself, a condensation of a
piece in William Seabrook's \textit{Dr. Wood Modern Wizard of the Laboratory}
(Harcourt Brace) 1941.  I chose this piece in response to the reports
of the discovery ``cold fusion'' in Utah.]
\nocite{wood:Nrays}
\nocite{seabrook:wood}



% The References section for this chapter is produced in ../example-ch01.tex

