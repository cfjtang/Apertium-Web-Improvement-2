% File freebmt09.tex
% July 23rd 2009
% Contact: freerbmt@dlsi.ua.es

\documentclass[11pt]{article}
\usepackage{freerbmt09}
\usepackage[utf8x]{inputenc}
\usepackage{times}
\usepackage{natbib}
\usepackage{url}
\usepackage{latexsym}

%\bibpunct{(}{)}{;}{A}{,}{,}
%\bibdata{mybib.bib}

\title{Reuse of Free Resources in Machine Translation between Norwegian Nynorsk and Bokmål}

\author{Jane Doe\\
  Department of Computer Science \\
  Nonesuch State University \\
  Utopia, NS 12345 \\
  {\tt jane.doe@cs.nsu.edu} \And
  John Smith \\
  Department of Linguistics \\
  Another State University \\
  Collegetown, AS 98765 \\  
  {\tt jsmith@ling.asu.edu}}

\date{}

\begin{document}

\maketitle

\begin{abstract}
  This article has a very long title, which should probably be snappier and more enticing since we want people to read the abstract to find out what it's really all about.
\end{abstract}

\section{Introduction}
The Oslo-Bergen Tagger \citep{hagen2000cbt}

\section{Design}
Apertium is a shallow-transfer and so on and so forth.

\section{Development}
Ie. the GsoC process? 

\section{Evaluation}
WER, coverage, qualitative assessment, anything else?

\section{Discussion}
Including future wossnames.

\section*{Acknowledgements}

Development was funded as part of the Google Summer of Code\footnote{\url{http://code.google.com/soc/}} programme.

\bibliographystyle{apalike}
\bibliography{apertium}


\end{document}
