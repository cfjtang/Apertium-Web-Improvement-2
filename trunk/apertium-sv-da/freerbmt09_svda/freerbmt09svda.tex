%%<spectie> PWER is position-independent word error rate
%%<spectie> 3--4 is for you please :)
%%<spectie> we need to make the evaluation
%%<spectie> could you do that?

% File freebmt09.tex
% July 23rd 2009
% Contact: freerbmt@dlsi.ua.es

\documentclass[11pt]{article}
\usepackage{freerbmt09}
\usepackage[utf8x]{inputenc}
\usepackage{times}
\usepackage{natbib}
\usepackage{url}
\usepackage{footnote}
\usepackage{latexsym}
\usepackage[small,bf]{caption}

%\bibpunct{(}{)}{;}{A}{,}{,}
%\bibdata{mybib.bib}

\newcommand{\com}[1]{
   \begin{quote}
     \begin{small}
       \begin{sf}\textbf{
         #1
       }\end{sf}
     \end{small}
   \end{quote}
} 

%\newcommand{\com}[1]{}

\title{Shallow-transfer rule-based machine translation for Swedish to Danish}

\author{Michael Kristensen\\
  Faculty of Pharm. Sci., \\
  University of Copenhagen,\\
  DK-2100 København, Denmark \\
  {\tt mkrist88@gmail.com} \And
  Francis M. Tyers\\
  Dept. Lleng. i Sist. Informàtics, \\
  Universitat d'Alacant,\\
  Alacant. E-03070\\  
  {\tt ftyers@dlsi.ua.es} \And
  Jacob Nordfalk\\
  Center for Videreuddannelse\\
  Ingeniørhøjskolen i København,\\
  DK-2750 Ballerup, Denmark\\
  {\tt jano@ihk.dk}}
%%  Ingeniørhøjskolen i København\\
%%  {\tt jacob.nordfalk@gmail.com}}
  
\date{}

\begin{document}

\maketitle

\begin{abstract}
  This article describes the development of a shallow-transfer machine translation
  system from Swedish to Danish in the Apertium platform. It gives details of the 
  resources used, the methods for constructing the system and an evaluation of the 
  translation quality.
\end{abstract}

\section{Introduction}

Both Swedish and Danish languages were standardised in the 12th to 15th centuries out
of the Old Norse which was spoken across Scandinavia. Swedish was standardised on the 
speech of the zone around Stockholm, where Danish was standardised on the speech of 
Copenhagen and surrounding areas. The languages are largely mutually intelligible \citep{Haugen1990}.

Given this, a machine translation system between the two languages should largely
focus on \emph{dissemination}, that is the production of text to be post-editted and 
published, rather than the production of text for \emph{assimilation}, or understanding.

This paper is laid out as follows, first a brief review is given of the design of a 
shallow transfer based machine translation system in the Apertium platform, we then 
dedicate a section to describing how the data for this system was created. Following this, 
an evaluation is given of the quality of the output of the system and its suitability 
for the post-edition task, and finally we finish with a discussion and prospectives 
for future work.

%% Some history of sv and da
%% previous research -- e.g. other sv->da systems
%% why it is desirable (need for publications in different standards, e.g. norden.org)
%% wide knowledge of English, less impulse to translate

\section{Design}

%% Apertium translation model

The Apertium platform follows a transfer-based machine translation model. A source
language text is first morphologically analysed using finite-state transducers. It is 
then disambiguated for part of speech by a bigram HMM part-of-speech tagger. Subsequently
lexical transfer is performed by the same module that performs syntactic transfer. Syntactic
transfer consists of matching fixed-length patterns of lexical units\footnote{A lexical unit 
is a lemma and its part of speech and morphological information.} and performing operations
such as insertions, removals and substitutions, along with concordancing. Figure~\ref{fig:apertium}
shows the main modules of the a given system built upon the platform.

Two models of syntactic transfer are supported by the platform, both single-stage, where only
one set of transfer rules is used, and three-stage where transfer rules are also used to 
group words into \emph{chunks}, on which later operations can be performed. The Swedish--Danish
pair uses the original, single stage transfer as a result of the closeness of the languages.


\begin{figure*} {\small \setlength{\tabcolsep}{0.5mm}
\begin{center}
\begin{tabular}{cccccccc}
\\
\parbox{1.2cm}{SL text} $\rightarrow$ & \framebox{\parbox{1.7cm}{deformatter}} \\ 
& $\downarrow$ \\
& \framebox{\parbox{1.3cm}{morph. analyser}}  $\rightarrow$ & \framebox{\parbox{1.1cm}{POS tagger}} $\rightarrow$ & \framebox{\parbox{1.3cm}{structural\ transfer}} $\rightarrow$ & \framebox{\parbox{1.5cm}{morph. generator}}  $\rightarrow$ & \framebox{\parbox{1.3cm}{post\-generator}} & & \\
& \parbox{1.3cm}{~} & \parbox{1.1cm}{~}  & $\updownarrow$ & & $\downarrow$ & &   \\
& \parbox{1.3cm}{~} & \parbox{1.1cm}{~}  & \framebox{\parbox{1.3cm}{lexical\ transfer}} & & \framebox{\parbox{1.6cm}{reformatter}} \\
& & & & &  $\downarrow$ \\
& & & & &  \parbox{1.2cm}{TL text}\\\\
\end{tabular}
\end{center} }
\caption{The eight modules of the shallow-transfer machine translation system}
\label{fig:apertium}
\end{figure*}

\section{Development}

\subsection{Resources}

% aspell-da and aspell-sv (for full-form lists)
% Extract (Cite)
% Mention the corpora ? 
% For Swedish mention Wiktionary.
% DSSO

We were able to make use of a number of freely available sources of information for 
constructing the system. Both Swedish and Danish have free, high coverage spell-checkers
available in the \emph{aspell}\footnote{\url{http://aspell.net/}} project. These were used
to provide lists of valid word forms for input into the \emph{Extract} \citep{Forsberg2006} 
program, which attempts to generate matches between lemmas and inflectional paradigms
based on full-form lists and extraction paradigms. 

The Swedish, Danish and English Wiktionaries have a fair amount of information regarding
both inflectional forms and translations between the two languages. 

\subsection{Analysis and generation}

%% Transducers built from scratch using above resources

The morphological dictionaries for the two languages were built in slightly 
different ways due to the differing amounts of information available. In both
cases however the closed categories were described manually. 

For Swedish, the Swedish Wiktionary\footnote{\url{http://sv.wiktionary.org}} has
inflectional tables for many of the words. In order to make use of these, all of 
the pages in a particular category (for example Nouns) were downloaded in HTML
form. The inflectional information was extracted using a variety of scripts and 
then all the possible paradigms were generated using the {\tt\small speling-tools}.\footnote{\url{http://wiki.apertium.org/wiki/Speling_tools}}
That is, for each word, one paradigm was created. These were then merged using
the same tools such that for each paradigm, any duplicates were removed. %% maybe improve this

Along with this, we also used \emph{Den stora svenska ordlistan} (DSSO),\footnote{\url{http://dsso.se}} 
a free full-form lexicon of Swedish, and entries extracted from a the {\tt\small aspell} spelling
dictionary of Swedish using the \emph{extract} tool \citep{Forsberg2006}.

\subsection{Disambiguation}

%% Train tagger using target language training

For disambiguation we first chose to train a basic unsupervised bigram part-of-speech tagger
using the {\tt\small apertium-tagger}. Although both tagged corpora and constraint
grammars exist for both Swedish and Danish, neither the constraint grammars
nor the corpora are free. The training corpora used were the 
Danish\footnote{\url{http://da.wikipedia.org}} and Swedish\footnote{\url{http://sv.wikipedia.org}; 
Access date: 8th February 2009} Wikipedias respectively.

Later, we used the target language training module in Apertium (as described in \cite{sanchez2008})
to train a Swedish tagger.

\subsection{Lexical transfer}

Despite the closeness of the languages, one of the most labour intensive parts of 
the work on this pair was the creation of the bilingual dictionary (transfer 
lexicon). Swedish and Danish are largely mutually intelligible, so there is not 
much demand for general purposes bilingual dictionaries between the two.

In order to create a dictionary we used several methods. The closed categories, 
for example pronouns, determiners, prepositions were added by hand, along with some 
of the open categories. Then, the following semi-automatic methods were used:

\begin{itemize} 
  \item Cognates -- The most obvious method for creating bilingual dictionary entries
     was to look at words which were the same in the two languages, or the same with
     different orthography. Frequent changes from Swedish to Danish include \emph{ö $\rightarrow$ ø}
     and \emph{ä $\rightarrow$ æ}. But also, non-orthographic changes such as verb endings
     in \emph{-a} in Swedish changing to \emph{-e} in Danish.
  \item Wordlists -- We came across a number of free untagged Swedish--Danish wordlists. In order
     to reuse this information, we first tagged both sides, and created new bilingual
     dictionary entries where both the part-of-speech and (in the case of nouns) the gender
     matched up. 
  \item Wiktionary -- Along with the previously mentioned inflection tables, the Swedish and 
     English Wiktionaries both have translations from Swedish to Danish. These were mined and 
     treated in a similar way to the wordlists above.
  \item Wikipedia -- For proper names, toponyms etc., we used the method described in \cite{tyers2008}
     to extract translations from Wikipedia.
  \item Probabilistic dictionary -- Finally we trained a statistical machine translation system
     using Moses \citep{Koehn2007} on the Europarl \cite{Koehn2005} corpus. From this we took
     the probabilistic lexicon, and performed the same operation as with the wordlists above. In doing
     this we simply took the most probable translation that was in both the Swedish and Danish 
     monolingual dictionaries.
\end{itemize}

All bilingual dictionary entries were manually checked and bad entries altered or discarded.

\subsection{Syntactic transfer}

As Swedish and Danish are closely-related languages, there are few translation problems
on the syntactic level. We created 17 transfer rules to deal with a number of divergences 
between the two languages. These were principally motivated by:

\begin{itemize}
  \item  Double definiteness -- In most definite NPs in Swedish, both the determiner \emph{den}
    and the definite form of the noun are used. In Danish when the determiner \emph{den} is 
    present, the definite form of the noun cannot be used. Compare in Swedish \emph{Den stora utmaningen}
    `The old challenge' with Danish \emph{Den store udfordring} `The old challege'.

  \item  Swedish supine verb form -- Swedish has a verb form called the supine which can be used with
    or without an auxiliary and functions somewhat like a past participle. Danish does not have this 
    verbal form, and in its place, often just uses a past participle, for example in Swedish 
    \emph{Han hade blivit trott} `He had been believed' translated to Danish \emph{Han var blevet troet} 
    `He was being believed' (literally, `Two people are been').
    Currently the auxiliary, required in Danish but sometimes omitted in Swedish (\emph{Han blivit trott}) is not being added.
  \item  Changes in auxiliary verbs -- There are some verbs in Swedish which do not take the same 
    auxiliary verb in forming periphrastic verb forms as in Danish, for example in Swedish \emph{Två 
    personer har förts} `Two people have been' translated to Danish \emph{To personer er ført} `Two people 
    have been' (literally, `Two people are been') 
  \item  Changes in passive formation -- In Swedish certain verbs in the passive (\emph{slå} `hit', 
    \emph{ligga} `lie', \emph{anta} `suppose', \ldots) must be translated in Danish using an inflected 
    form of the verb \emph{blive} `become' in the active voice and the past participle.
\end{itemize}

Other changes made in the transfer rules include changing a passive followed by an infinitive in Swedish
to passive followed by full infinitive in Danish, for example in Swedish \emph{Tros ha dödat} `Believed to have
killed' is expressed in Danish as \emph{Menes at have dræbt}.

\subsection{Status}

The table in table~\ref{table:status} gives details of the current status of the system
in terms of the number of lemmas in each of the dictionaries and the number of transfer rules.

\begin{table}
\centering
\begin{tabular}{|l|c|}
\hline
                                           & Number entries\\
\hline
Monolingual dict. ({\tt sv})               & 5,230 lemmas \\
Bilingual dict.                            & 6,854 lemmas \\
Monolingual dict. ({\tt da})               & 10,694 lemmas \\
Transfer rules ({\tt sv $\rightarrow$ da}) & 17 rules \\
\hline
\end{tabular}
    \caption{Status of pair as of ...}
    \label{table:status}
\end{table}

\section{Evaluation}

The evaluation was split into four parts, the first is an evaluation of the coverage
of the system with respect to a number of available corpora of Swedish. The second
provides a quantitative evaluation using post-edition word error rate (WER) which 
gives an indication as to how much work a post-editor needs to do in order to 
achieve an adequate target language translation. The third is a qualitative evaluation
which looks at some of the major deficiencies of the system with respect to disambiguation,
and lexical and syntactic transfer. Finally we provide a short comparative evaluation of
our system against two proprietary systems.

\subsection{Coverage}

The vocabulary coverage of the system is calculated over two available corpora. Here coverage
is defined as \emph{na\"ive coverage}, that is for any given surface form at least one analysis
is returned. This may not be complete. The first corpus is a database dump of the Swedish 
Wikipedia, the second is the Swedish sentences from the EuroParl corpus \cite{Koehn2005}. The 
results are presented in table~\ref{table:coverage}.

\begin{table*}
\centering
\begin{tabular}{|l|c|c|c|}
\hline
Corpus & Running tokens & Known tokens & Coverage \\
\hline
Wikipedia  & 30,662,861 & 22,030,690 & 71.84\%\\
EuroParl   & 15,531,107 & 12,499,971 & 80.48\%\\
\hline
\end{tabular}
    \caption{Naïve coverage for two corpora}
    \label{table:coverage}
\end{table*}

\subsection{Quantitative}

The quantitative evaluation involved the post-edition of 65 machine translated
sentences (1151 words) from the Swedish Wikipedia. The sentences were selected from an 
article on history, run through the Apertium machine translation system and then a human
post-editor corrected the resulting Danish translation.

%% See http://wiki.apertium.org/wiki/Swedish_and_Danish/Evaluation
\begin{table}
\centering
\begin{tabular}{|l|c|c|c|}
\hline
Corpus    & WER & PWER & Free rides\\
\hline
Wikipedia & 31 \% & 28 \%  & 38 \% \\
\hline
\end{tabular}
    \caption{Evaluation results for the post-edition task}
    \label{table:quanteval}
\end{table}

Both word error rate (WER) and position-independent error rate (PWER) were 
calculated by counting the number of insertions, substitutions and deletions
between the post-editted text and the original translation.

The results of this evaluation are shown in table~\ref{table:quanteval} and 
indicate that the system is still not ready for being used in a post-edition 
environment, where to correct a text with a word error rate of over 15\% takes
longer than translation from scratch.
%% Francis: 15% ? How do you know? Reference.

\subsection{Qualitative}



%% Principle downfalls -- treatment of supine when not written with auxilliary
%% No support for compounds -- even if both words in the compound are in the dict.
%% Low coverage -- what percentage of errors are due to unknown words ?
%% NEED EXAMPLES FOR THESE

\subsection{Comparative}

%% Google, Gramtrans, Apertium -- 100 sentences each

There are two existing proprietary machine translation systems online which translate 
between Swedish and Danish, Gramtrans\footnote{\url{http://www.gramtrans.com}} and Google
Translate\footnote{\url{http://translate.google.com}}. Gramtrans is a rule-based system
built on top of constraint grammar, and Google Translate is a statistical machine translation
system. To compare the results of the three systems, we used the post-edited Danish text as a
reference and compared translations for all of the systems as in the quantitative evaluation.
This gives a small bias in favour of Apertium which must be taken into account when making conclusions.

%% system. To compare the results of the three systems, we used the selected 100 sentences and translated
%% them with all of the systems, they were then post-editted and the WER and PWER calculated
%% as in the quantitative evaluation. The results of this evaluation are shown in table~\ref{table:compeval}
%% and show that FILL IN HERE

The results of this evaluation are shown in table~\ref{table:compeval} and show that, in spite 
of the natural bias favouring the Apertium translation, Gramtrans scores better and Google 
Translate does much worse than Apertium.

Here is an example of the translations

Original:
"Det finns en kort överfart vid det baltiska havet vid Helsingborg, 
på vilket ställe Själland kan ses från Skåne, ett vanligt tillhåll för vikingar."

Apertium:
"Det findes en kort *överfart ved det *baltiska havet ved Helsingborg, 
på hvilket *ställe *Själland kan ses fra Skåne, et *vanligt tilhold før vikinger."

Postedited:
"Der findes en kort overfart ved det baltiske hav ved Helsingborg, 
på hvilket sted Sjælland kan ses fra Skåne, et sædvanligt tilholdssted for vikinger."

Gramtrans:
"Der findes en kort overfart ved det baltiske hav ved Helsingborg, 
på hvilket sted Själland kan ses fra Skåne, et sædvanligt tilhold for vikinger."

Google:
"Der er en kort passage i Østersøen i Helsingborg, i hvilken plads 
Zealand kan ses fra Scania, en regelmæssig tilholdssted for vikingerne."


Google Translate is using English as inter-language and therefore some English artifacts are present, mostly proper nouns and time indications. But worse the meaning of the text is sometimes harder to percieve in Google Translate than in the original text.



\begin{table}
\centering
\begin{tabular}{|l|c|c|c|}
\hline
System & Edit distance & WER & PWER\\
\hline
Apertium  & 353   & 31    & 28 \\
Gramtrans & 308   & 26    & 20 \\
Google    & 557   & 48    & 33 \\
\hline
\end{tabular}
    \caption{Comparative evaluation results for 65 sentences}
    \label{table:compeval}
\end{table}

\section{Discussion}

We have presented results from the first free-software translator of Swedish to Danish. This
is also the first translator between two Germanic languages to be released as part of the 
Apertium platform. 

In terms of future work, we intend to reverse the direction to also translate from Danish to 
Swedish, to improve the vocabulary coverage, and to improve part-of-speech disambiguation. There
is a free constraint grammar for Norwegian \citep{hagen2000cbt} available, that could, with some
conversion work be altered to work as a constraint grammar for Danish (Norwegian Bokmål is even
closer to Danish than Swedish is to Danish). Finally, the transfer rules could be expanded to 
deal with the cases where a supine is used without auxiliary, and a method of handling compound
words could be implemented.

%% Perhaps incorporate info from SALDO (cite) -- wasn't known about until after development over

\section*{Acknowledgements}

Development was funded as part of the Google Summer of Code\footnote{\url{http://code.google.com/soc/}} 
programme. Many thanks to Thyge Larsen for his assistance with post-edition and evaluation.

\bibliographystyle{apalike}
\bibliography{freerbmt09svda}

\end{document}
