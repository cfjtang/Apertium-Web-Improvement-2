\section{Evaluation}
% Hrvatski -> Slovenski
\begin{table}
\begin{tabular}{lrr}

\textbf{Language} & \textbf{SETimes} & \textbf{Europarl}\\
\hline
Serbo-Croatian & 85.41\% & -- \\
Slovenian &  -- & 95.50\%\\
\hline
\end{tabular}
\caption{ Naïve coverage }
\label{table:coverage}
\end{table}


This sections covers the evaluation of the developed system. 
The system was tested by measuring the lexical coverage, and by performing
a qualitative and a quantitative evaluation. 

Lexical coverage tested using existing free corpora, 
while the quantitative evaluation was performed on 100 postedited sentences from the Slovenian news portal 
Delo \footnote{\url{http://www.delo.si/}}.


\subsection{Lexical coverage}

Coverage for the Serbo-Croatian--Slovenian language pair was measured using both the SETimes \citep{tyers2010south} and Europarl \citep{koehn05a} corpora. 
We measured coverage naively, meaning that we assume a word is in our 
dictionaries if at least one of its surface forms is found in the corpus. 
We are aware of the shortcomings of such an evaluation framework, 
however we decided to use it because of its simplicity.

The Serbo-Croatian $\rightarrow$ Slovenian side was evaluated using the SETimes corpus.
As SETimes does not cover Slovenian
the Slovenian $\rightarrow$ Serbo-Croatian side was evaluated only on the EuroParl corpus. The results are shown in table \ref{table:coverage}.

\subsection{Quantitative}

The quantitative evaluation was performed by 5 articles
from the Slovenian news portal Delo.
The articles were translated from slovenian using Apertium, and were later corrected by a human post-editor in order to get a correct translationn.
The word error rate (WER) was calculated
by counting the number of insertions, substitutions and deletions between the post-edited articles
and the original translation. We used the freely available apertium-eval-translator for calculating the WER 
and for bootstrap resampling \cite{koehn04}.
We also reported the percentage of out of vocabulary words (OOV), and the total number of words per article.
The results are given in table \ref{table:quantitative1}.

We also calculated both metrics for the output of Google Translate\footnote{\url{http://translate.google.com/}} 
and the results are presented in the same tables. 

Given all the assumptions the WER and PER matrics make, the results show that our system is comparable to Google Translate
with regards to translation quality. The Slovenian $\rightarrow$ Serbo-Croatian translation seems to be better than the Serbo-Croatian $\rightarrow$ Slovenian one
which is due to the fact that more effort was put into developing the former direction.


% cp /path/to//apertium//trunk/apertium-eval-translator/WER.pl /path/to/bin
% perl apertium//trunk/apertium-eval-translator/bootstrap_resampling.pl -n 1000 -e /path/to//apertium//trunk/apertium-eval-translator/wer.sh -s SOURCE_FILE -t TEST_FILE -r REF_FILE

\begin{table*}
\begin{center}
\begin{tabular}{|l|r|rr||rr|}
   \hline
  \multirow{2}{*}{\textbf{Article}}  & \multirow{2}{*}{\textbf{Words}} & \multicolumn{2}{|c||}{\textbf{\%~OOV}} & \multicolumn{2}{c|}{\textbf{WER}}\\\cline{3-6}
                    &                & Apertium & Google &  Apertium & Google \\
   \hline
   \hline
  \texttt{maraton}  & 243            & 16.8     & --     & \textbf{[42.85, 47.92]} & [64.39, 74.56] \\
  \texttt{sonce}    & 169            & 17.7     & --     & \textbf{[32.65, 45.33]}    & [47.27, 58.62] \\
  \texttt{merkator} & 414            & 16.9     & --     & \textbf{[38.78, 48.14]}     & [56.13, 70.30] \\
  \texttt{volitve}  & 229            & 13.9     & --     & [37.81, 53.36]      & [46.66, 62.67] \\
%  \texttt{jame}     & 994            & 14.9     & --     & ??      & [57.28, 67.65] \\
                                                           % 49.57	
  \hline
  \hline
  \texttt{maraton}  & 245            & 37.7     & --     & [52.78, 56.25]           & [45.58, 63.87]\\
  \texttt{sonce}    & 171            & 17.5     & --     & [47.50, 62.79]    & [32.10, 58.49] \\
  \texttt{merkator} & 424            & 12.9     & --     & [45.78, 56.56]    & [48.46, 64.15] \\
  \texttt{volitve}  & 226            & 16.8     & --     & [47.00, 58.44]    & [38.09, 58.10]\\
%  \texttt{jame}     & 1,005          & 15.9     & --     & [??, ??]              & [??,??]\\

  \hline
\end{tabular}
 \caption{Results for Word Error Rate (WER) in the Slovenian$\rightarrow$Serbo-Croatian direction (top) and Serbo-Croatian$\rightarrow$Slovenian (bottom). Scores in bold show a statistically significant improvement over the other system according to bootstrap resampling at $p = 0.95$.}
\label{table:quantitative1}
\end{center}
\end{table*}




\subsection{Qualitative}
The biggest problems are currently caused by the incompleteness of our dictionaries.
The issues caused by OOV words are twofold.
The less important issue is the fact that the system is unable to provide a translation for the unknown words. 
However, the more important issue is that OOV words cause problems with disambiguation and transfer, since they
break long chains of words into smaller ones and drastically reduce context information. 

Next, we have seen that the number of disambiguation rules for Slovenian is not sufficient for high quality disambiguation. 
The constraint grammar for the Slovenian side was written based on the constraint grammar for the Serbo-Croatian side,
and it needs further work.

We have also noticed difficulties in the transfer because of the loose grammar of both sides.
Adding additional rules does not significantly improve the performance of the system 
and OOV words make long transfer rules irrelevant.

Finally, beacause of the small timeframe, we were not able to work much on lexical selection.
Our lexical selection module is the least developed part of our system. 
We have not done any work on the Slovenian side and the number of rules for the Serbo-Croatian side is small.
This is due to the fact that no reliable parallel corpus exists for this language pair.



