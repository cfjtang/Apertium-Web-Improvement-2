%
% File mtsxiv.tex, a copy of eamt12.tex
%
% Contact: mlf@dlsi.ua.es

%%% To ease future customizations, various replaceables have been paramaterized
%%% as listed in the newcommands section

\documentclass[11pt]{article}
\usepackage{mtsxiv}
\usepackage{times}
\usepackage{latexsym}
\usepackage{url}
\setlength\titlebox{6.5cm}    % Expanding the titlebox
%%% YOUR PACKAGES BELOW THIS LINE %%%


\newcommand{\confname}{Machine Translation Summit XIV}
\newcommand{\website}{\protect\url{http://www.mtsummit2013.info/}}
\newcommand{\contactname}{research track co-chair Mikel L.\ Forcada}
\newcommand{\contactemail}{mlf@dlsi.ua.es} 
\newcommand{\conffilename}{mtsxiv}
\newcommand{\downloadsite}{\protect\url{http://www.mtsummit2013.info/}}
\newcommand{\paperlength}{$8$ (eight)}
\newcommand{\shortpaperlength}{$4$ (four)}

\newcommand{\todo}[1]{\{\textbf{TODO: #1}\}}
\newcommand{\note}[1]{\{\textbf{Note: #1}\}}

\title{Shallow-transfer rule-based machine translation for the Western group of South Slavic languages}

\author{Hrvoje Peradin\\
  Affiliation / Address line 1\\
  Affiliation / Address line 2\\
  Affiliation / Address line 3\\
  {\tt email@domain}  \And
  Francis M. Tyers\\
  Dept. Lleng. i Sist. Inform.\\
  Universitat d'Alacant\\
  E-03899 Alacant\\
  {\tt ftyers@dlsi.ua.es}}

\date{}

\begin{document}
\maketitle
\begin{abstract}
  This paper describes the development of a bidirectional machine
  translation system for the western branch of South-Slavic languages
  (comprised of Serbo-Croatian and Slovene). Details of resources and
  development methods used are given, as well as an evaluation, and
  general directives for future work \note{this is just copied}.
\end{abstract}

\section{Credits}

\todo{Credit where credit is due...}

\section{Introduction}

\todo{On Slovene}

\todo{On Serbo-Croatian}

\begin{figure}

(?) Venn diagram of Serbo-Croatian, Serbian, Croatian, Bosnian, Montenegrin, 
Neo-Štokavian, Čakavian, Kajkavian, Torlakian
Ekavian, Ijekavian, Ikavian

\end{figure}

\todo{On their relationship}

\section{Design}

\subsection{The Apertium platform}
\nocite{forcada2011apertium}
The Apertium\footnote{\url{http://wiki.apertium.org/}}  platform follows a modular machine translation model.
Morphological analysis of the source text is performed by a letter 
transducer compiled from a morphological lexicon,\footnote{A morphological lexicon 
contains ordered pairs of word surface forms and their lemmatised analyses.} 
and cohorts\footnote{A cohort 
consists of a surface form and one or more readings containing the lemma of the 
word and the morphological analysis.} obtained in this manner go through
a disambiguation process. 
Disambiguated readings proceed to a bilingual dictionary also performed 
by a letter transducer and  through a two-level syntactic transfer, 
which performs word reordering, deletions, insertions, and basic syntactic chunking.
The final module is a letter transducer that generates surface forms in the target language from 
the bilingual transfer output.

\subsection{Constraint Grammar}
The 
%initial 
disambiguation in this language pair is performed by a
Constraint Grammar (CG) module\footnote{Implemented in the CG3 formalism, using the \texttt{vislcg3} compiler, available under GNU GPL. For a detailed reference see: \url{http://beta.visl.sdu.dk/cg3.html}}. CG is a 
paradigm that uses hand-written rules to reduce the problem 
of linguistic ambiguity. A series of context-dependent rules are applied 
to a stream of tokens and readings for a given surface form are excluded, 
selected or assigned additional tags.

\section{Development}

\subsection{Morphological analysis and generation}

\subsection{Disambiguation}

\subsection{Lexical selection}

\subsection{Transfer}

\begin{table}

 Dict statistics sh, sl, sh-sl
\end{table}

\begin{table}

\begin{tabular}{lrr}
               & sh$\rightarrow$sl & sl$\rightarrow$sh \\
\hline
Disambiguation &     194              &     27 \\
Lexical selection &                   &  \\
Transfer &                   &  \\

\end{tabular}
 \caption{Statistics on the number of rules in each direction}
\end{table}

\section{Evaluation}

\subsection{Lexical coverage}

\subsection{Quantitative}

\subsection{Qualitative}


\section{Conclusions}


\section*{Acknowledgements}

Filip, Jernej, Ale\v{s}

% \bibliography{\confname}

\begin{thebibliography}{}

\bibitem[\protect\citename{Aho and Ullman}1972]{Aho:72}
Aho, Alfred~V. and Jeffrey~D. Ullman.
\newblock 1972.
\newblock {\em The Theory of Parsing, Translation and Compiling}, volume~1.
\newblock Prentice-{Hall}, Englewood Cliffs, NJ.

\bibitem[\protect\citename{{American Psychological Association}}1983]{APA:83}
{American Psychological Association}.
\newblock 1983.
\newblock {\em Publications Manual}.
\newblock American Psychological Association, Washington, DC.

\bibitem[\protect\citename{{Association for Computing Machinery}}1983]{ACM:83}
{Association for Computing Machinery}.
\newblock 1983.
\newblock {\em Computing Reviews}, 24(11):503--512.

\bibitem[\protect\citename{Chandra \bgroup et al.\egroup }1981]{Chandra:81}
Chandra, Ashok~K., Dexter~C. Kozen, and Larry~J. Stockmeyer.
\newblock 1981.
\newblock Alternation.
\newblock {\em Journal of the Asso\-ciation for Computing Machinery},
  28(1):114--133.

\bibitem[\protect\citename{Gledson and Keane}2008a]{Gledson:08homog}
Gledson, Anne, and John Keane. 
\newblock 2008a. 
\newblock Measuring Topic Homogeneity and its Application to Dictionary-Based Word-Sense Disambiguation. 
\newblock {\em Coling 2008, 22nd International Conference on Computational Linguistics}, Manchester, UK.
\newblock 273--280.
\bibitem[\protect\citename{Gledson and Keane}2008b]{Gledson:08websearch}
Gledson, Anne, and John Keane. 
\newblock 2008b. 
\newblock Using Web-Search Results to Measure Word-group Similarity. \newblock {\em Coling 2008, 22nd International Conference on Computational Linguistics}, Manchester, UK.
\newblock 281--288.
\bibitem[\protect\citename{Gusfield}1997]{Gusfield:97}
Gusfield, Dan.
\newblock 1997.
\newblock {\em Algorithms on Strings, Trees and Sequences}.
\newblock Cambridge University Press, Cambridge, UK.

\bibitem[\protect\citename{Tam and Schultz}2006]{Tam:06}
Tam, Yik-Cheung and Tanja Schultz.
\newblock 2006. 
\newblock Unsupervised Language Model Adaptation Using
  \nobreak{Latent} Semantic Marginals.
\newblock {\em Interspeech 2006 -- ICSLP, Ninth International Conference on Spoken Language Processing}, 
Pittsburgh, Pennsylvania, paper 1705-Thu1A2O.2. 

\bibitem[\protect\citename{Tam and Schultz}2007]{Tam:07}
Tam, Yik-Cheung and Tanja Schultz.
\newblock 2007.
\newblock Correlated \nobreak{Latent} Semantic Model for
  Unsupervised Language Model Adaptation.
\newblock {\em Proceedings of ICASSP 2007, International Conference on 
  Acoustics, Speech, and Signal Processing}, Honolulu, Hawaii, Vol. IV, 41--44.

\end{thebibliography}

\end{document}
