%
% File mtsxiv.tex, a copy of eamt12.tex
%
% Contact: mlf@dlsi.ua.es

%%% To ease future customizations, various replaceables have been paramaterized
%%% as listed in the newcommands section

\documentclass[11pt]{article}
\usepackage{mtsxiv}
\usepackage{times}
\usepackage{latexsym}
\usepackage{url}
\setlength\titlebox{6.5cm}    % Expanding the titlebox
%%% YOUR PACKAGES BELOW THIS LINE %%%
\usepackage[utf8]{inputenc}

\newcommand{\confname}{Machine Translation Summit XIV}
\newcommand{\website}{\protect\url{http://www.mtsummit2013.info/}}
\newcommand{\contactname}{research track co-chair Mikel L.\ Forcada}
\newcommand{\contactemail}{mlf@dlsi.ua.es} 
\newcommand{\conffilename}{mtsxiv}
\newcommand{\downloadsite}{\protect\url{http://www.mtsummit2013.info/}}
\newcommand{\paperlength}{$8$ (eight)}
\newcommand{\shortpaperlength}{$4$ (four)}

\usepackage{multibib}
\usepackage[comma]{natbib}
\usepackage{chapterbib}

\newcommand{\todo}[1]{\{\textbf{TODO: #1}\}}
\newcommand{\note}[1]{\{\textbf{Note: #1}\}}

\title{Shallow-transfer rule-based machine translation for the Western group of South Slavic languages}

\author{Hrvoje Peradin\\
  Affiliation / Address line 1\\
  Affiliation / Address line 2\\
  Affiliation / Address line 3\\
  {\tt email@domain}  \And
  Filip Petkovski \\
  Affiliation / Address line 1\\
  Affiliation / Address line 2\\
  Affiliation / Address line 3\\
  {\tt email@domain}  \And
  Francis M. Tyers\\
  Dept. Lleng. i Sist. Inform.\\
  Universitat d'Alacant\\
  E-03899 Alacant\\
  {\tt ftyers@dlsi.ua.es}}

\date{}

\begin{document}
\maketitle
\begin{abstract}
  This paper describes the development of a bidirectional machine
  translation system for the western branch of South-Slavic languages
  (comprised of Bosnian, Serbian, Croatian and Slovene). Details of
  resources and development methods used are given, as well as an
  evaluation, and general directives for future work.
\end{abstract}

\section{Credits}

\todo{Credit where credit is due...}

\section{Introduction}

\todo{On Slovene}

\todo{On Serbo-Croatian}

\begin{figure}

\todo{Venn diagram of Serbo-Croatian, Serbian, Croatian, Bosnian, Montenegrin, 
Neo-Štokavian, Čakavian, Kajkavian, Torlakian
Ekavian, Ijekavian, Ikavian}

\end{figure}

\todo{On their relationship}

\section{Design}

\subsection{The Apertium platform}
\nocite{forcada2011apertium}
The Apertium\footnote{\url{http://wiki.apertium.org/}} platform is a
modular machine translation system. The typical core layout consists
of a letter transducer morphological lexicon\footnote{A list of
ordered pairs of word surface forms and their lemmatised
analyses.}. The transducer produces cohorts\footnote{A cohort 
consists of a surface form and one or more readings containing the lemma of the 
word and the morphological analysis.} which are then subjected to a
morphological disambiguation process.
%
Disambiguated readings are then run through a second letter transducer
containing bilingual dictionary information, and through a two-level
syntactic transfer system that performs word reordering, deletions,
insertions, and basic syntactic chunking.
%
The final module is yet another letter transducer which generates
surface forms in the target language from the bilingual transfer
output cohorts.

\subsection{Constraint Grammar}
This language pair uses a Constraint Grammar (CG)
module\footnote{Implemented in the CG3 formalism, using the
  \texttt{vislcg3} compiler, available under GNU GPL. For a detailed
  reference see: \url{http://beta.visl.sdu.dk/cg3.html}} for
disambiguation. The CG formalism consists of hand-written rules that
are applied to a stream of tokens. Depending on the morphosyntactic
context of a given token the rules select or exclude readings of a
given surface form, or assign additional tags.

\section{Development}

\subsection{Resources}
This language pair was developed with the
aid of on-line resources containing word definitions and flective
paradigms, such as \emph{Hrvatski jezični
  portal}\footnote{\url{http://hjp.srce.hr}} for the BCMS side. For
the Slovene side we used a similar online resource \emph{Slovar
  slovenskega knjižnega
  jezika},\footnote{\url{http://bos.zrc-sazu.si/sskj.html}}, and the
\emph{Amebis Besana} flective
lexicon.\footnote{\url{http://besana.amebis.si/pregibanje/}}

The bilingual dictionary for the language pair was developed from scratch,
using the \emph{EUDict}\footnote{\url{http://eudict.com/}} online
dictionary and \emph{Google
  Translate}\footnote{\url{http://translate.google.com/}}.

\subsection{Morphological analysis and generation}
The basis for this language pair are the morphological
lexicons for BCMS (from
the language pair BCMS-Macedonian, {\small{\tt apertium-bhs-mk}}) and Slovene (from the
language pair Slovene-Spanish, {\small{\tt apertium-sl-es}}). Both
lexicons are written in the XML formalism of
\emph{lttoolbox}\footnote{\url{http://wiki.apertium.org/wiki/Lttoolbox}}
(\citealp{rojas2005construccion}), and were developed as parts of
their respective language pairs, during the Google Summer of Code
2011.\footnote{\url{http://code.google.com/soc/}}. Since the lexicons
had been developed using different frequency lists, and slightly
different tagsets, they have been further trimmed and updated to
synchronise their coverage.

%wiktionaries and Wikipedia, as well as an SETimes corpus\footnote{\url{http://opus.lingfil.uu.se/SETIMES.php}} (\citealp{tyers2010south}) and a%corpus composed from the Serbian, Bosnian, Croatian and Serbo-Croatian Wikipedias.

%% Other resources for morphological analysis of Serbian and Croatian exist
%% (\citealp{vitas2004intex}, \citealp{vitas2003processing}, \citealp{agic2008improving}, \citealp{snajder08automatic}), 
%% to our knowledge there are none freely available for either Serbian, Bosnian or
%% Croatian. 

\subsection{Disambiguation}
Though for both languages there exists a number of tools for
morphological taggin and disambiguation (\todo{reference some}), there
are none freely available. Likewise, the adequatly tagged corpora are
mostly non-free (\todo{reference some non-free corpora, like
  1984.,...}). \todo{See on the taggers/corpora for Slovene}\todo{See
  on the taggers/corpora for BCMS}.
Since both BCMS Slovene are highly flective languages, the
automatically trained tagger would be inefficient blabla, we chose to omit the
statistical tagger canonically used in Apertium language pairs, and
use solely Constraint Gramar (CG) for disambiguation. The CG module
does not provide complete disambiguation, so in the case of any
remaining ambiguity the system picks the first output analysis.

\todo{Disambiguation examples}

\subsection{Lexical transfer}
The lexical transfer was done with an \emph{lttoolbox} letter
transducer composed of bilingual dictionary entries. Additional
paradigms were added to the transducer to compensate for the tagset
notational differences.

\subsection{Lexical selection}
\todo{Lexical selection examples}

\todo{use some sh\_HR and sh\_SR examples perhaps 'univerza' $\rightarrow$ \{univerzitet,sveučilište\}}


\subsection{Transfer}
\todo{describe apertium transfer}
The BCMS and Slovene languages are very closely related, and their
morphologies are extremely similar. Most of non-technical transfer
rules are thus written only for rare syntactic differences. These are
mostly about clitic ordering, and different noun case usage.

\todo{Transfer examples, esp. the differences}

\begin{table}

\begin{tabular}{lrrr}
\textbf{Dictionary} & \textbf{Paradigms} & \textbf{Entries} & \textbf{Forms} \\
\hline
Serbo-Croatian &  - & - & - \\
Slovenian &  - & - & - \\
\hline
Bilingual &  - & - & - \\
\hline
\end{tabular}
\caption{ Dict statistics sh, sl, sh-sl}

\end{table}

\begin{table}

\begin{tabular}{lrr}
               & sh$\rightarrow$sl & sl$\rightarrow$sh \\
\hline
Disambiguation &     194              &     27 \\
Lexical selection &                   &  \\
Transfer &                   &  \\

\end{tabular}
 \caption{Statistics on the number of rules in each direction}
\end{table}

\section{Evaluation}

\subsection{Lexical coverage}

\subsection{Quantitative}

\subsection{Qualitative}


\section{Conclusions}


\section*{Acknowledgements}

Jernej, Ale\v{s}

%\nocite{*}
\bibliographystyle{mtsxiv}
\bibliography{mtsxivslsh}
\end{document}

%% \begin{thebibliography}{}

%% \bibitem[\protect\citename{Aho and Ullman}1972]{Aho:72}
%% Aho, Alfred~V. and Jeffrey~D. Ullman.
%% \newblock 1972.
%% \newblock {\em The Theory of Parsing, Translation and Compiling}, volume~1.
%% \newblock Prentice-{Hall}, Englewood Cliffs, NJ.

%% \bibitem[\protect\citename{{American Psychological Association}}1983]{APA:83}
%% {American Psychological Association}.
%% \newblock 1983.
%% \newblock {\em Publications Manual}.
%% \newblock American Psychological Association, Washington, DC.

%% \bibitem[\protect\citename{{Association for Computing Machinery}}1983]{ACM:83}
%% {Association for Computing Machinery}.
%% \newblock 1983.
%% \newblock {\em Computing Reviews}, 24(11):503--512.

%% \bibitem[\protect\citename{Chandra \bgroup et al.\egroup }1981]{Chandra:81}
%% Chandra, Ashok~K., Dexter~C. Kozen, and Larry~J. Stockmeyer.
%% \newblock 1981.
%% \newblock Alternation.
%% \newblock {\em Journal of the Asso\-ciation for Computing Machinery},
%%   28(1):114--133.

%% \bibitem[\protect\citename{Gledson and Keane}2008a]{Gledson:08homog}
%% Gledson, Anne, and John Keane. 
%% \newblock 2008a. 
%% \newblock Measuring Topic Homogeneity and its Application to Dictionary-Based Word-Sense Disambiguation. 
%% \newblock {\em Coling 2008, 22nd International Conference on Computational Linguistics}, Manchester, UK.
%% \newblock 273--280.
%% \bibitem[\protect\citename{Gledson and Keane}2008b]{Gledson:08websearch}
%% Gledson, Anne, and John Keane. 
%% \newblock 2008b. 
%% \newblock Using Web-Search Results to Measure Word-group Similarity. \newblock {\em Coling 2008, 22nd International Conference on Computational Linguistics}, Manchester, UK.
%% \newblock 281--288.
%% \bibitem[\protect\citename{Gusfield}1997]{Gusfield:97}
%% Gusfield, Dan.
%% \newblock 1997.
%% \newblock {\em Algorithms on Strings, Trees and Sequences}.
%% \newblock Cambridge University Press, Cambridge, UK.

%% \bibitem[\protect\citename{Tam and Schultz}2006]{Tam:06}
%% Tam, Yik-Cheung and Tanja Schultz.
%% \newblock 2006. 
%% \newblock Unsupervised Language Model Adaptation Using
%%   \nobreak{Latent} Semantic Marginals.
%% \newblock {\em Interspeech 2006 -- ICSLP, Ninth International Conference on Spoken Language Processing}, 
%% Pittsburgh, Pennsylvania, paper 1705-Thu1A2O.2. 

%% \bibitem[\protect\citename{Tam and Schultz}2007]{Tam:07}
%% Tam, Yik-Cheung and Tanja Schultz.
%% \newblock 2007.
%% \newblock Correlated \nobreak{Latent} Semantic Model for
%%   Unsupervised Language Model Adaptation.
%% \newblock {\em Proceedings of ICASSP 2007, International Conference on 
%%   Acoustics, Speech, and Signal Processing}, Honolulu, Hawaii, Vol. IV, 41--44.

%% \end{thebibliography}


