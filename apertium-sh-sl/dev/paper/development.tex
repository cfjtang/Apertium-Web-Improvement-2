\section{Development}

\subsection{Resources}
This language pair was developed with the
aid of on-line resources containing word definitions and flective
paradigms, such as \emph{Hrvatski jezični
  portal}\footnote{\url{http://hjp.srce.hr}} for the BCMS side. For
the Slovene side we used a similar online resource \emph{Slovar
  slovenskega knjižnega
  jezika},\footnote{\url{http://bos.zrc-sazu.si/sskj.html}}, and the
\emph{Amebis Besana} flective
lexicon.\footnote{\url{http://besana.amebis.si/pregibanje/}}

The bilingual dictionary for the language pair was developed from scratch,
using the \emph{EUDict}\footnote{\url{http://eudict.com/}} online
dictionary and \emph{Google
  Translate}\footnote{\url{http://translate.google.com/}}.

\subsection{Morphological analysis and generation}
The basis for this language pair are the morphological
lexicons for BCMS (from
the language pair BCMS-Macedonian, {\small{\tt apertium-bhs-mk}}) and Slovene (from the
language pair Slovene-Spanish, {\small{\tt apertium-sl-es}}). Both
lexicons are written in the XML formalism of
\emph{lttoolbox}\footnote{\url{http://wiki.apertium.org/wiki/Lttoolbox}}
(\citealp{rojas2005construccion}), and were developed as parts of
their respective language pairs, during the Google Summer of Code
2011.\footnote{\url{http://code.google.com/soc/}}. Since the lexicons
had been developed using different frequency lists, and slightly
different tagsets, they have been further trimmed and updated to
synchronise their coverage.

%wiktionaries and Wikipedia, as well as an SETimes corpus\footnote{\url{http://opus.lingfil.uu.se/SETIMES.php}} (\citealp{tyers2010south}) and a%corpus composed from the Serbian, Bosnian, Croatian and Serbo-Croatian Wikipedias.

%% Other resources for morphological analysis of Serbian and Croatian exist
%% (\citealp{vitas2004intex}, \citealp{vitas2003processing}, \citealp{agic2008improving}, \citealp{snajder08automatic}), 
%% to our knowledge there are none freely available for either Serbian, Bosnian or
%% Croatian. 

\subsection{Disambiguation}
Though for both languages there exists a number of tools for
morphological taggin and disambiguation (\todo{reference some}), there
are none freely available. Likewise, the adequatly tagged corpora are
mostly non-free (\todo{reference some non-free corpora, like
  1984.,...}). \todo{See on the taggers/corpora for Slovene}\todo{See
  on the taggers/corpora for BCMS}.  Since both BCMS Slovene are
highly inflected languages, the automatically trained statistical
tagger canonically used in Apertium language pairs would not give
satisfactory results. For this reason we chose to use solely
Constraint Gramar (CG) for disambiguation. The CG module does not
provide complete disambiguation, so in the case of any remaining
ambiguity the system picks the first output analysis.

Due to the similarities between the languages, we were able to
reuse much of the rules developed earlier for BCMS. Following are
some examples of disambiguation rules:

\begin{itemize}
\item Preposition based case disambiguation

\sentenceexample{
Za našo ljubo staro mater. $\leftrightarrow$ Za našu dragu staru majku.

[For{\sc.pr.acc}] [our{\sc.prn.acc}] [dear{\sc.prn.acc}] [old{\sc.prn.acc}] [mother{\sc.n.acc}]

(For our dear old mother.)
}

Noun phrases in both languages typically generate a great number of
ambiguities. For this phrase the morphological analyser gives:

{\small
\begin{Verbatim}
    "<Za>"
         "za" pr acc 
    ;    "za" pr gen
    ;    "za" pr ins
    "<našo>"
         "naš" prn pos p1 f sg acc 
    ;    "naš" prn pos p1 f sg ins
    "<ljubo>"
         "ljub" adj f sg acc ind 
    ;    "ljubo" adv sint
    ;    "ljub" adj nt sg nom ind
    ;    "ljub" adj nt sg acc ind
    ;    "ljub" adj f sg ins ind
    "<staro>"
         "star" adj f sg acc ind 
    ;    "staro" adv sint
    ;    "star" adj f sg ins ind
    ;    "star" adj nt sg nom ind
    ;    "star" adj nt sg acc ind
"<mater>"
         "mati" n f sg acc 
    ;    "mati" n f du gen
    ;    "mati" n f pl gen
\end{Verbatim} 
}


\end{itemize}



\subsection{Lexical transfer}
The lexical transfer was done with an \emph{lttoolbox} letter
transducer composed of bilingual dictionary entries. Additional
paradigms were added to the transducer to compensate for the tagset
notational differences.

\subsection{Lexical selection}

Since there was no adequate and free Slovene -- Serbo-Croatian parallel corpus, 
we chose to do the lexical selection relying only on Apertium's lexical selection module.
For cases not covered by our hand-written rules, the system would choose the lexical 
default from the bilingual dictionary.
We provide examples of such lexical selection rules.

Phonetics based lexical selection: many words from the Croatian and Serbian dialects differ in a single phoneme.
An example are the words \emph{točno} in Croatian and \emph{tačno} in Serbian (engl. \emph{accurate}).
Such differences were solved through the lexical selection module using rules like:

{\small
\begin{Verbatim}
<rule>
    <match lemma="točno" tags="adv.*">
	<select lemma="točno" tags="adv.*"/>
    </match>
</rule>
\end{Verbatim}
}
for Croatian, and
{\small
\begin{Verbatim}
<rule>
    <match lemma="točno" tags="adv.*">
	<select lemma="tačno" tags="adv.*"/>
    </match>
</rule>
\end{Verbatim}
}
for Serbian and Bosnian.

Similarly, the Croatian language has the form \emph{burza} (meaning stock exchange in English), while Serbian and Bosnian have \emph{berza}. 
For those forms the following rules were written:

{\small
\begin{Verbatim}
<rule>
    <match lemma="borza" tags="n.*">
        <select lemma="burza" tags="n.*"/>
    </match>
</rule>
\end{Verbatim}
}
for Croatian, and 
{\small
\begin{Verbatim}
<rule>
    <match lemma="borza" tags="n.*">
	<select lemma="berza" tags="n.*"/>
    </match>
</rule>

\end{Verbatim}
}
for Serbian and Bosnian.

Another example of a phonetical difference are words which have h in Croatian and Bosnian, but v in Serbian.
Such words include \emph{kuha} and \emph{duhan} in Croatian and Bosnian, but \emph{kuva} and \emph{duvan} in Serbian.
Similar rules were written for the forms for \emph{porcelain} (procelan in Serbian and porculan in Croatian), 
\emph{salt} (so and sol) etc.

While the Serbian dialect accepts the Ekavian and Ikavian reflexes, 
the Croatian dialect uses only the Ijekavian reflex.
Since the selection for the different reflexes of the yat vowel is done in the generation process,
no rules were needed in the lexical selection module.

Internationalisms have been introduced to Croatian and Bosnian mainly through the Italian and German language
whereas they have entered Serbian through French and Russian. 
As a result, the three dialects have developed different phonetic patterns for internatonal words.

Examples of rules for covering such varieties include:
{\small
\begin{Verbatim}
<rule>
    <match lemma="Betlehem" tags="np.*">
	<select lemma="Betlehem" tags="np.*"/>
    </match>
</rule>
\end{Verbatim}
}
for Croatian and Bosnian, and
{\small
\begin{Verbatim}
<rule>
    <match lemma="Betlehem" tags="np.*">
	<select lemma="Vitlejem" tags="np.*"/>
    </match>
</rule>
\end{Verbatim}
}
for Serbian.

Finally, the Croatian months used for the Gregorian calendar have Slavic-derived names and differ from the original Latin names.
For example, the Croatian language has the word \emph{siječanj} for \emph{January}, and 
the Serbian language has the word \emph{Januar}.
These differences were also covered by the lexical selection module.

Besides the yat reflex, several other cases were not covered in the lexical selection module. These include the pronoun "what" which has the form "što" in Croatian and the forms "što" and "šta" in Serbian and Bosnian, depending on whether the context is interrogative or relative. This ambiguity was left out of the lexical selection module and was dealth with during generation.

In Croatian with modal verbs the infinitive is prescribed while in Serbian the construct da + present is usually used.
For example the Croatian translation of the sentence "I want to eat" is "Želim \emph{jesti}", while the Serbian translation is "Želim \emph{da jedem}". This difference was not covered in the lexical selection module.






\subsection{Lexical selection}
\todo{Lexical selection examples}

\todo{use some sh\_HR and sh\_SR examples perhaps 'univerza' $\rightarrow$ \{univerzitet,sveučilište\}}

\subsection{Transfer}

The BCMS and Slovene languages are very closely related, and their
morphologies are extremely similar. Most of non-technical transfer
rules are thus written only for rare syntactic differences. These are
mostly about clitic ordering, and different noun case usage.

Following are examples of transfer rules, which also illustrate some
contrastive characteristics of the languages:

\begin{itemize}
\item The future tense:
\enumsentence{

Ja ću gledati\footnote{The encliticised future tense forms (gledat ću / gledaću) are handled equally.} $\rightarrow$ Jas bom gledal

[I] [will{\sc.clt.p1.sg}] [watch{\sc.inf}] $\rightarrow$ [I] [will{\sc.clt}] [watch{\sc.pres.p1.sg}]

(I will watch.)
}
\end{itemize}

