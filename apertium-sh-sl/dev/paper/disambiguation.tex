\subsection{Disambiguation}
Though for both languages there exists a number of tools for
morphological taggin and disambiguation (\todo{reference some}), there
are none freely available. Likewise, the adequatly tagged corpora are
mostly non-free (\todo{reference some non-free corpora, like
  1984.,...}). \todo{See on the taggers/corpora for Slovene}\todo{See
  on the taggers/corpora for BCMS}.  Since both BCMS Slovene are
highly inflected languages, the automatically trained statistical
tagger canonically used in Apertium language pairs would not give
satisfactory results. For this reason we chose to use solely
Constraint Gramar (CG) for disambiguation. The CG module does not
provide complete disambiguation, so in the case of any remaining
ambiguity the system picks the first output analysis.

Due to the similarities between the languages, we were able to
reuse much of the rules developed earlier for BCMS. Following are
some examples of disambiguation rules:

\begin{itemize}
\item Preposition based case disambiguation

\sentenceexample{
Za našo ljubo staro mater. $\leftrightarrow$ Za našu dragu staru majku.

[For{\sc.pr.acc}] [our{\sc.prn.acc}] [dear{\sc.prn.acc}] [old{\sc.prn.acc}] [mother{\sc.n.acc}]

(For our dear old mother.)
}

Noun phrases in both languages typically generate a great number of
ambiguities. For this phrase the morphological analyser gives:

{\small
\begin{Verbatim}
    "<Za>"
         "za" pr acc 
    ;    "za" pr gen
    ;    "za" pr ins
    "<našo>"
         "naš" prn pos p1 f sg acc 
    ;    "naš" prn pos p1 f sg ins
    "<ljubo>"
         "ljub" adj f sg acc ind 
    ;    "ljubo" adv sint
    ;    "ljub" adj nt sg nom ind
    ;    "ljub" adj nt sg acc ind
    ;    "ljub" adj f sg ins ind
    "<staro>"
         "star" adj f sg acc ind 
    ;    "staro" adv sint
    ;    "star" adj f sg ins ind
    ;    "star" adj nt sg nom ind
    ;    "star" adj nt sg acc ind
"<mater>"
         "mati" n f sg acc 
    ;    "mati" n f du gen
    ;    "mati" n f pl gen
\end{Verbatim} 
}


\end{itemize}

