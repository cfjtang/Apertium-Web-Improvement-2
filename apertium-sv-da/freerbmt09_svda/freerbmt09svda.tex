% File freebmt09.tex
% July 23rd 2009
% Contact: freerbmt@dlsi.ua.es

\documentclass[11pt]{article}
\usepackage{freerbmt09}
\usepackage[utf8x]{inputenc}
\usepackage{times}
\usepackage{natbib}
\usepackage{url}
\usepackage{footnote}
\usepackage{latexsym}
\usepackage[small,bf]{caption}


%\bibpunct{(}{)}{;}{A}{,}{,}
%\bibdata{mybib.bib}

\newcommand{\com}[1]{
   \begin{quote}
     \begin{small}
       \begin{sf}\textbf{
         #1
       }\end{sf}
     \end{small}
   \end{quote}
} 

%\newcommand{\com}[1]{}

\title{Shallow-transfer rule-based machine translation for Swedish to Danish}

\author{Michael Kristensen\\
  Dept. Computer Science, \\
  Bangladesh University of Eng. and Tech.,\\
  Bangladesh, NS 12345 \\
  {\tt jane.doe@cs.buet.ac.bd} \And
  Francis M. Tyers\\
  Dept. Lleng. i Sist. Informàtics, \\
  Universitat d'Alacant,\\
  Alacant. E-03070\\  
  {\tt ftyers@dlsi.ua.es} \And
  Jacob Nordfalk\\
  {\tt foo@bar}}
  

\date{}

\begin{document}

\maketitle

\begin{abstract}
  This article ...
\end{abstract}

\section{Introduction}

\section{Design}

\section{Development}

\subsection{Resources}

\subsection{Analysis and generation}

\subsection{Disambiguation}

%% Train tagger using target language training

\subsection{Lexical transfer}

Despite the closeness of the languages, one of the most labour intensive parts of 
the work on this pair was the creation of the bilingual dictionary (transfer 
lexicon). Swedish and Danish are largely mutually intelligible, so there is not 
much demand for general purposes bilingual dictionaries between the two.

In order to create a dictionary we used several methods. The closed categories, 
for example pronouns, determiners, prepositions were added by hand, along with some 
of the open categories. Then, the following semi-automatic methods were used:

\begin{itemize} 
  \item Cognates --
  \item Wordlists --
  \item Wiktionary --
  \item Wikipedia --
  \item Probabilistic dictionary --
\end{itemize}

\subsection{Syntactic transfer}

\section{Evaluation}

The evaluation was split into four parts, the first is an evaluation of the coverage
of the system with respect to a number of available corpora of Swedish. The second
provides a quantitative evaluation using post-edition word error rate (WER) which 
gives an indication as to how much work a post-editor needs to do in order to 
achieve an adequate target language translation. The third is a qualitative evaluation
which looks at some of the major deficiencies of the system with respect to disambiguation,
and lexical and syntactic transfer. Finally we provide a short comparative evaluation of
our system against two proprietary systems.

\subsection{Coverage}

%% Lexical coverage over Wikipedia / General corpus of Swedish.

\subsection{Quantitative}

\subsection{Qualitative}

%% Principle downfalls -- treatment of supine when not written with auxilliary

\subsection{Comparative}

%% Google, Gramtrans, Apertium -- 100 sentences each

\section{Discussion}

%% Future work: Reverse the direction, more scandinavian pairs, 
%% increase vocabulary coverage, improve disambiguation -- possibly try and convert OB tagger for nb to Danish
%% Perhaps incorporate info from SALDO (cite) -- wasn't known about until after development over
%% Compound handling

\cite{hagen2000cbt}

\section*{Acknowledgements}

Development was funded as part of the Google Summer of Code\footnote{\url{http://code.google.com/soc/}} programme.

\bibliographystyle{apalike}
\bibliography{freerbmt09svda}


\end{document}
