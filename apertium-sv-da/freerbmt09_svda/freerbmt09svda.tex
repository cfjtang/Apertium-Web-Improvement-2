% File freebmt09.tex
% July 23rd 2009
% Contact: freerbmt@dlsi.ua.es

\documentclass[11pt]{article}
\usepackage{freerbmt09}
\usepackage[utf8x]{inputenc}
\usepackage{times}
\usepackage{natbib}
\usepackage{url}
\usepackage{footnote}
\usepackage{latexsym}
\usepackage[small,bf]{caption}

%\bibpunct{(}{)}{;}{A}{,}{,}
%\bibdata{mybib.bib}

\newcommand{\com}[1]{
   \begin{quote}
     \begin{small}
       \begin{sf}\textbf{
         #1
       }\end{sf}
     \end{small}
   \end{quote}
} 

%\newcommand{\com}[1]{}

\title{Shallow-transfer rule-based machine translation for Swedish to Danish}

\author{Michael Kristensen\\
  Dept. Computer Science, \\
  University of Eng. and Tech.,\\
  Denmark, NS 12345 \\
  {\tt jane.doe@cs.buet.ac.bd} \And
  Francis M. Tyers\\
  Dept. Lleng. i Sist. Informàtics, \\
  Universitat d'Alacant,\\
  Alacant. E-03070\\  
  {\tt ftyers@dlsi.ua.es} \And
  Jacob Nordfalk\\
  {\tt foo@bar}}
  

\date{}

\begin{document}

\maketitle

\begin{abstract}
  This article ...
\end{abstract}

\section{Introduction}

%% Some history of sv and da
%% previous research -- e.g. other sv->da systems
%% why it is desirable (need for publications in different standards, e.g. norden.org)
%% wide knowledge of English, less impulse to translate

\section{Design}

%% Apertium translation model

\section{Development}

\subsection{Resources}

% aspell-da and aspell-sv (for full-form lists)
% Extract (Cite)
% Mention the corpora ? 
% For Swedish mention Wiktionary.
% DSSO

\subsection{Analysis and generation}

%% Transducers built from scratch using above resources

The morphological dictionaries for the two languages were built in slightly 
different ways due to the differing amounts of information available. In both
cases however the closed categories were described manually. 

For Swedish, the Swedish Wiktionary\footnote{\url{http://sv.wiktionary.org}} has
inflectional tables for many of the words. In order to make use of these, all of 
the pages in a particular category (for example Nouns) were downloaded in HTML
form. The inflectional information was extracted using a variety of scripts and 
then all the possible paradigms were generated using the {\tt\small speling-tools}.\footnote{\url{...}}
That is, for each word, one paradigm was created. These were then merged using
the same tools such that for each paradigm, any duplicates were removed. %% maybe improve this

\subsection{Disambiguation}

%% Train tagger using target language training

For disambiguation we first chose to train a basic unsupervised bigram part-of-speech tagger
using the {\tt\small apertium-tagger}. Although both tagged corpora and constraint
grammars exist for both Swedish and Danish, neither the constraint grammars
nor the corpora are free.

Later, we used the target language training module in Apertium (as described in \cite{sanchez2008})
to train a Swedish tagger.

\subsection{Lexical transfer}

Despite the closeness of the languages, one of the most labour intensive parts of 
the work on this pair was the creation of the bilingual dictionary (transfer 
lexicon). Swedish and Danish are largely mutually intelligible, so there is not 
much demand for general purposes bilingual dictionaries between the two.

In order to create a dictionary we used several methods. The closed categories, 
for example pronouns, determiners, prepositions were added by hand, along with some 
of the open categories. Then, the following semi-automatic methods were used:

\begin{itemize} 
  \item Cognates --
  \item Wordlists --
  \item Wiktionary --
  \item Wikipedia -- 
  \item Probabilistic dictionary --
\end{itemize}

\subsection{Syntactic transfer}

%% Double definiteness
%% Supine (just convert to a pp.-- förts → ført )
%% Passives -- blive etc.
%% Some auxiliaries are different. (Två personer har förts → To personer er ført )
%%   Done using a list

\subsection{Status}

\begin{table}
\centering
\begin{tabular}{|l|c|}
\hline
                                           & Number entries\\
\hline
Monolingual dict. ({\tt sv})               & 5,230 lemmas \\
Bilingual dict.                            & 6,854 lemmas \\
Monolingual dict. ({\tt da})               & 10,694 lemmas \\
Transfer rules ({\tt sv $\rightarrow$ da}) & 17 rules \\
\hline
\end{tabular}
    \caption{Status of pair as of ...}
    \label{table:status}
\end{table}

\section{Evaluation}

The evaluation was split into four parts, the first is an evaluation of the coverage
of the system with respect to a number of available corpora of Swedish. The second
provides a quantitative evaluation using post-edition word error rate (WER) which 
gives an indication as to how much work a post-editor needs to do in order to 
achieve an adequate target language translation. The third is a qualitative evaluation
which looks at some of the major deficiencies of the system with respect to disambiguation,
and lexical and syntactic transfer. Finally we provide a short comparative evaluation of
our system against two proprietary systems.

\subsection{Coverage}

%% Lexical coverage over Wikipedia / General corpus of Swedish.

\subsection{Quantitative}

\begin{table}
\centering
\begin{tabular}{|l|c|c|c|}
\hline
Corpus    & WER & PWER & Free rides\\
\hline
Wikipedia & -   & -    & -\\
\hline
\end{tabular}
    \caption{Evaluation results for the post-edition task}
    \label{table:quanteval}
\end{table}


\subsection{Qualitative}

%% Principle downfalls -- treatment of supine when not written with auxilliary
%% No support for compounds -- even if both words in the compound are in the dict.

\subsection{Comparative}

%% Google, Gramtrans, Apertium -- 100 sentences each

\begin{table}
\centering
\begin{tabular}{|l|c|c|c|}
\hline
System & WER & PWER & Free rides\\
\hline
Apertium  & -   & -    & -\\
Google    & -   & -    & -\\
Gramtrans & -   & -    & -\\
\hline
\end{tabular}
    \caption{Comparative evaluation results for 100 sentences}
    \label{table:compeval}
\end{table}

\section{Discussion}

%% Future work: Reverse the direction, more scandinavian pairs, 
%% increase vocabulary coverage, improve disambiguation -- possibly try and convert OB tagger for nb to Danish
%% Perhaps incorporate info from SALDO (cite) -- wasn't known about until after development over
%% Compound handling

\cite{hagen2000cbt}

\section*{Acknowledgements}

Development was funded as part of the Google Summer of Code\footnote{\url{http://code.google.com/soc/}} programme.

\bibliographystyle{apalike}
\bibliography{freerbmt09svda}

\end{document}
