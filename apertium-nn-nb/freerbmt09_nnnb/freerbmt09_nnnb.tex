% Created 2009-09-06 Sun 21:10
\documentclass[11pt]{article}
\usepackage{freerbmt09}
\usepackage[utf8x]{inputenc}
\usepackage{times}
\usepackage{natbib}
\usepackage{url}
\usepackage{latexsym}

\usepackage{hyperref}
\author{Jane Doe\\  Department of Computer Science \\  Nonesuch State University \\  Utopia, NS 12345 \\  {\tt jane.doe@cs.nsu.edu} \And  John Smith \\  Department of Linguistics \\  Another State University \\  Collegetown, AS 98765 \\    {\tt jsmith@ling.asu.edu}}

\title{Reuse of Free Resources in Machine Translation between Norwegian Nynorsk and Bokmål}
%\author{Kevin Brubeck Unhammer}
\date{06/09, 2009}

\begin{document}

\maketitle


\begin{abstract}

  This article has a very long title, which should probably be snappier and more enticing since we want people to read the abstract to find out what it's really all about.
\end{abstract}

\section{Introduction}
\label{sec-1}

The term \emph{Norwegian} covers a variety of related spoken dialects. Up
until the 1800's, Danish was the only written standard used in
Norway. Bokmål emerged through various reforms which brought the
written language closer to the spoken; Nynorsk however, was created
from the ground up with the purpose of representing all the spoken
dialects of Norway. As it is, certain dialects (especially around the
Oslo area) correspond more with Bokmål, while others are closer to
Nynorsk. Nynorsk is ``in a minority position in Norway, with
approximately 12\% of the users'' \citep{everson2000sln}, or around
450,000 people. 

We describe a machine translation system between Nynorsk and Bokmål
built using the Apertium platform \citep{corbi05oss}\ldots{}

Although Nynorsk is in a minority position, there are quite good
linguistic resources for both Nynorsk and Bokmål available under Free
licences.

The Oslo-Bergen Tagger \citep{hagen2000cbt} \ldots{}


On the Scandinavian language group, and expanding it for Apertium:
\begin{quote}
Morphologically, these four languages are equally distant from each
other, but the terminological differences are smaller between Nynorsk
and Bokmål than between the other two. \\
\citep{everson2000sln}
\end{quote}






\section{Design}
\label{sec-2}

The nn-nb module follows the design of the Apertium
system\citep{corbi05oss}, a shallow-transfer pipeline system. Apertium is
quite modular, and nn-nb differs from most of the other language pairs
in using a Constraint Grammar morphological disambiguator.
\section{Development}
\label{sec-3}

\begin{itemize}
\item exact matches ``adding exact matches if morphology is the same
  e.g. if a noun is (sg.ind, pl.ind, sg.def, pl.def) in both
  languages and is spelt the same then added to the bidix''
\item matches with differences a lot of bidix entries were created by
  1.finding nb entries w/o translations

\begin{enumerate}
\item running some replacements for typical differences in substrings
\item checking whether the altered entries were in nn
\end{enumerate}

\item Giza++ (I guess I could do a diff on the bidix from before and after
  I started working on Giza++ stuff)
\end{itemize}
\section{Evaluation}
\label{sec-4}

We define naïve coverage as the proportion of words in a corpus which
are given at least one analysis by our monolingual
dictionaries. Testing on Nynorsk Wikipedia (5116174 words) and Bokmål
Wikipedia (27529115 words), we have 89.6\% and 88.2\% coverage,
respectively.

The Word Error Rate (WER) on a 3750 word Wikipedia article on
linguistics in the Bokmål to Nynorsk direction was 22.06\% when
including unknown words, although since 64.93\% of these were
free-rides (ie. the same in Bokmål and Nynorsk) anyway, the final WER
was 10.71\%. Typical free-rides include names, loan-words and special
terminology.

\begin{itemize}
\item Qualitative assessment\ldots{}
\item anything else?
\item Anything about Nyno? (Their web page says 74000 words, don't know
  about WER but the cool thing about Nyno is the interface, ie. the
  freedom of choice with variants and how the user can do the lexical
  selection. The examples from
  \href{http://www.hf.uio.no/tekstlab/Presseklipp/Spr%E5knytt%203-2001.htm}{http://www.hf.uio.no/tekstlab/Presseklipp/Spr\%E5knytt\%203-2001.htm}
  (``Nyno i bruk'') seem to indicate that the OBT is a bit better at
  disambiguating though (underline meaning wrong translation):

\begin{itemize}
\item Original: Når det iverksettes arbeidskamp, er det partene i den
    enkelte tvist som har ansvaret for de konsekvenser arbeidskamp
    påfører tredjemann.
\item Nyno: Når \underline{den iverksettes arbeidskampen}, er \underline{dei} partane i den
    enkelte tvist som har ansvaret for dei konsekvensane arbeidskamp
    påfører tredjemann.
\item Apertium: Når det blir iverksett *arbeidskamp, er det partane i
    den einskilde tvisten som har ansvaret for dei konsekvensane
    *arbeidskamp påfører tredjemann.
\end{itemize}

\end{itemize}
\section{Discussion}
\label{sec-5}

\begin{itemize}
\item We don't have any sort of compound handling, probably we could
  analyse a whole lot more with a trie or whatever, but there's also a
  compound handler in OBT that might be possible to integrate.

\begin{itemize}
\item *menneskehandel.
\item menneske. handel.
\end{itemize}

\item ``Well-written'' nynorsk uses lots of periphrasis and MWE's, eg. particle
  verbs; we don't generate any such thing. A syntactic analysis might
  be useful here, although without being quite certain of where the
  relevant phrase ends, it'll be safer to stick with non-discontinuous
  target language translations.
\end{itemize}
\section{Acknowledgements}
\label{sec-6}

Development was funded as part of the Google Summer of Code\footnote{\href{http://code.google.com/soc/}{http://code.google.com/soc/} }
programme. Thanks to mentors and OBT people.

\bibliographystyle{apalike}
\bibliography{apertium}







\end{document}